\chapter{Analysis} \label{analysis}

As mentioned in the previous chapter section \ref{queries}, the
QuERIES inspired methodology is used in this chapters analysis. Tables
\ref{pomdbtable} and \ref{probtable} show relevant information about
the calculations. In table \ref{pomdbtable} is listed all the possible
states (S), defender actions (D), attacker actions (A) and
observations (O). Then, every transition from state to another state
is calculated as a probability. 

\section{Modeling the problem and quantifying the models} \label{modprob}

The example project of this thesis is an image showing system that
could be used e.g for advertisement, public transport timetables or
practically anywhere where static media should be presented. Our use
case focuses on displaying arbitrary media in a public location. The
results, if an an intruder should gain unauthorised access, would be
anywhere from displaying improper imagery, to succeeding in displaying
propaganda or other unwanted content. Unauthorised access could have a
negative economic effect for the service provider, as every
organisation displaying media want to remain credible among users.

The attack surface of the example project focuses on physical access
and vulnerabilities in remote connections. With MQTT-messaging, SSH
and display protocols, internal and external messaging takes place.

If a unauthorised access would happen, the results would probably
affect a part of the society, as the arbitrary content could gain
media and social media attention. This public humiliation would
definitely affect the credibility of the service provider, as well as
the customer. Possible propaganda could affect society by spreading
false information, or possibly bringing up societal issues via
activism. Either way, this would be unwanted from the perspective of
the service provider, customer and users.

As mentioned in chapter \ref{securitystandpoints} section
\ref{measuringsecurity} subsection \ref{queriesasmethodology}, the
value of intellectual property is defined as $\alpha$, for which other
parameters refer to as fractions of it. If this was applied to a real
setting, the value of intellectual property would be calculated appropriately
for the scenario.

\section{Modeling the possible attacks}

In this section, the possible attacks are modeled using an attack
chart, depicted as a POMDP, which is a modeling tool presented earlier. In the
original paper where QuERIES is presented, an important parameter is
the time to reverse-engineer the system without prior information
about the protection scheme \cite{carin2008cybersecurity}. Our
approach is different; the attacks are considered as successful, if
they gain further leverage in the attack graph, e.g transitioning
state ''idle'' to ''partial loss of system''. This is to maintain
cohesion in the study, as we don't need to define what it means to
''reverse engineer'' the system. In this case, information of the
system is publicly available as an open source project, thus the
information of the system being available also to the attacker.

In table \ref{pomdbtable}, the first column describes states the
system can be in. The second and the third column state defensive
actions, and observations of the system. The fourth column contains
template of the attacks that could be conducted. After forming the
attack graph depicted in the table, the attacker tries to breach the
system, leveraging through the A0-A4 column. The probabilities are
then calculated with the model presented in chapter
\ref{securitystandpoints} section \ref{quantitativemetrics} subsection
\ref{queries}.

The algorithm produces a reward value for each QuERIES iteration,
which is used to improve the setup. For calculating reward values, R
script with library ''pomdp'' was used.

\begin{landscape}
\begin{table}
\centering
\begin{tabular}{ |c|c|c|c| }
 \hline & & & & S0-S7 & D0-D4 & O0-O4 & A0-A4 \\ & & & & \hline \hline
 & & & & Idle & Monitor system & Normal operation & Intercept MQTT
 messaging \\ & & & & \hline & & & & Receive media through MQTT &
 Isolate system & Detected suspicious activity & Compromise Github
 repository \\ & & & & \hline & & & & Set up SystemD services &
 Shutdown system & Detected unauthorised access & Gain physical access to
 device \\ & & & & \hline & & & & Start Weston & Isolate device &
 Detected unusual media display & Exploit vulnerabilities in display
 \\ & & & & \hline & & & & Display media & & & Exploit vulnerabilities
 in SSH connections \\ & & & & \hline & & & & Partial loss of system & \ & \ &
 \\ & & & & \hline & & & & Complete loss of system & & & \\ & & & &
 \hline & & & & & & & \\ & & & & \hline

\end{tabular}
\caption{Different states, defensive measures, observations and attack
  measures for the system.}
\label{pomdbtable}
\end{table}
\end{landscape}

A weight of 1 was used for positive results, and -100, if something
was to be compromised. This weight distribution is due to the fact
that even if blue team succeeds most of time, the results of failure
are much worse than a succeeding result from the blue team. The
discount constant is used in calculations as shown in
\ref{pomdpappenix}, where it influences the priority of immediate
versus future rewards \cite{mcabeeMarkov}. Our case signifies the
importance of both, so value of 0.75 was used. Note that the maximum
reward value is 4, and the minimum is -400.

\section{Using the results} \label{usingtheresults}

The results are investigated through the probabilities, which are
represented in the table. The reward score is taken in account on how
successful/unsuccessful the setup is from a security perspective.

As stated by Hughes and Cybenko, using the results means evaluating
the gained results to decide if proposed protections are adequate for
our means \cite{hughes2013quantitative}. The QuERIES model was
iterated 2 times, and the results were placed in the following table.

\begin{table}[h!]
\centering
\begin{tabular}{|c|c|c|}
  \hline Iteration & Reward function & Proposed protections \\ \hline
  1 & -25 & Multiple \\ \hline
  2 & 4 & None
  \\ \hline
\end{tabular}
\caption{Protections applied in each iteration}
\label{iterationtable}
\end{table}

The reward function in the first iteration is calculated with the
script in appendix \ref{pomdpappendix}. The second value is calculated
with as the maximum accumulated points; the highest value is 4 points
without any discount, as the red team failed to provide any results
for the second iteration.

As seen, the reward function is growing as the proposed protections
are applied in each iteration. In the following figure we can see that
the time to breach peaks in the first hour.

\begin{figure}[t!]
\centerline{\includesvg[width=1.0\columnwidth]{../graph/barchart.svg}}
\caption{Probabilities to breach into the system.}
\label{timetobreach}
\end{figure}

The time distribution of breaches are depicted in the graph
\ref{timetobreach}. The chart has similar shape as the chart in
\cite{carin2008cybersecurity}, meaning that attacks in their initial
phase usually succeed more frequently.

The following chart is used to decide, when is the optimal time to
stop the attacks from the attacker perspective. We use this also to
reflect the cost estimate of the blue team. Following the methodology
in referred papers by Hughes and Cybenko, the optimal time to stop the
attacks is seen in the following graph using open and closed loop
strategies \cite{hughes2013quantitative}.

\begin{figure}[t!]
\centerline{\includesvg[width=1.0\columnwidth]{../graph/cost_graph.svg}}
\caption{The most optimal time to stop the attack using open and
  closed loop algorithms.}
\label{openandclosed}
\end{figure}

\subsection{Improving the system using Nix}

The following measures were taken in the first iteration: disable
USB ports, change MQTT password to token based authentication and change
SSH to key based authentication. As we can see, the QuERIES output
provided concrete results on what are the weak points of the
system. The reward function gives overall insight on how
secure/insecure the system is. This generalization is one of the
issues found in POMDP, as the reward function itself is too general to
taken seriously. 

As an example, the exposed USB-ports of the system were found to be a
problem, due to the possibility of infecting the system with direct
control to the system. This can be mitigated easily by modifying the
client devices configuration.nix, with the following changes:

\begin{figure}[H]
\begin{lstlisting} 
{ boot.kernelPackages = pkgs.linuxPackages_latest; boot.kernelParams =
  [ "nousb" # Disables USB at the kernel level ]; boot.kernelModules =
  [ ]; boot.extraModulePackages = [ ]; services.udev.extraRules = ''
  SUBSYSTEM=="usb", ACTION=="add", OPTIONS+="ignore_device" ''; }
\end{lstlisting}
\label{kernelsnippet}
\end{figure}

The configuration could be applied to any number of clients, proving that Nix could
be used to rapidly address arising security issues. This supports the
argument presented in chapter \ref{imperative}, that through
updatability the security could be improved. The configuration is only an example to mitigate hardware access attack
vectors, as there are many more ways for the attacker to leverage the
access, e.g by replacing the whole device, depending on the resources
of the attacker. This highlight the scalability of Nix systems; it
doesn't matter if we have one or thousands of devices, updating them
is equally simple. A huge contrast between imperative and
declarative systems is found, as imperative systems would need linear
administration time in relation with the number of devices. This is a
great benefit for system administrators.

\subsection{Issues with QuERIES} \label{issues}

One issue found using QuERIES with POMDP was that applying the
reward functions is in our case rather arbitrary. We assumed, that for
positive results, the reward function is defined as 1, and for the
negative outcomes -100. This is due to the negative results deemed as
catastrophic and the positive results being slightly positive. The
choice for both parameters could have been any integer, but the issue
is that it demands ''gut-feeling'' of the authors to select the
appropriate parameters. This could be mitigated with some study on
selecting the paramers, but it's definitely a challenge to provide
good foundation on such arbitrary numbers.

One remark using POMDP is that it produces very generalized
output. This is why we use methodologies such as QuERIES to improve
the system and the POMDP being just one component. Other benefit from
using POMDP is that it provides us concise attack graph, and I argue
that using POMDP may even be redundant. In POMDP, negative rewards
signify that the system has issues, and positive meaning that the
system is more secure \cite{mcabeeMarkov}. More ergonomic approach would
be calculating rewards without positive outcomes, due to them
overshadowing the most critical issues. Developing a pessimistic
model that highlights explicitly the weak points of a system would be
a topic for further research. 

\section{Generalizing the results}



The next chapter handles the topic for further research, inspired by
the applied usage of QuERIES.
