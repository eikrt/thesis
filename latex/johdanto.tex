\chapter{Introduction} \label{johdanto}

A Linux distribution is a bundle of the Linux kernel and a set of
software products called packages \cite{gnuPackagesx2014}. A package
manager is an instrument that handles building packages from either
from source or pre-built binaries, resolving build-time and run-time
dependencies of packages and installing, removing, and upgrading
packages in user environments \cite{gnuPackagesx2014}. Every Linux
device must handle its installed programs with their dependencies and
configurations either imperatively or declaratively. Overwhelmingly
large portion of Linux distributions fall in to the first category
\cite{dolstra2008nixos}. Both imperative and declarative systems have
their strengths and weaknesses from administrative and security
standpoints.

An imperative system provides updatability and modification through a
destructive instrument. Popular imperative package managers are apt,
apk, dnf and zypper \cite{dolstra2008nixos}. Imperative package
managers can remove and overwrite existing files which leaves the
system in an inconsistent state. Different installs have by nature
different states which causes many problems discussed in this
thesis. As files are cross-modified through packages with package
managers such as apt, upgrading can be disastrous as such systems
don't support atomic rollback capabilities \footnote{Some Linux
distributions using btrfs filesystem can perform a snapshot and
rollback \cite{opensuseSystemRecovery}}. Due to the unpredictability,
often the result can be a partially or completely broken system
\cite{dolstra2008nixos}.

The reference imperative system in this thesis is Debian with its
default package manager apt, due to it's popularity and relative
simplicity. The reference declarative system is
NixOS with it's partial namesake package manager Nix which provides declarative
configuration of the whole system including the Linux
kernel\footnote{There exists an experimental project that has
succeeded with BSD interoperability \cite{githubGitHubNixosbsdnixbsd}}. Nix is configured by the Nix
programming language which is inspired by purely functional languages
such as Haskell. \cite{van2013reference}

The reference architecture depicted in chapter \ref{architecture} is
based on NixOS as it is the most popular purely functional,
declarative Linux distribution with over 100 000 packages
\cite{nixosNixOSSearch}. Another good alternative would have been
Guix, which has over 28 000 packages \cite{gnuPackagesx2014}. Guix has
some improvements over Nix, including richer and more extensible
programming environment with a Lisp-dialect configuration language,
Scheme \cite{courtes2021deploiements}. NixOS remains as the
distribution of choice, as the number of packages is greater, and
general support is found to be better.

This thesis focuses on the systems and information security of a
reference architecture created with NixOS. Chapter \ref{architecture}
goes through a reference architecture of a solution that handles
securely the most critical functionalities of an image displaying system.

This thesis discusses how declarative systems can be used as an
improvement over imperative systems in a public information screen
setting. In chapter \ref{imperative} both approaches to package
management is compared, and in chapters \ref{architecture} and
\ref{analysis} a quantitative research is carried out revealing
strengths and weaknesses of the reference Nix environment
setup. Selected methodologies provide quantitative results which can
be used to improve the security of similar declarative
systems. Propositions for further research are gone through in chapter
\ref{further} and finally, chapter \ref{conclusion} concludes the
thesis.

\section{Research methodologies and questions}

Research methodologies used in this thesis are:
\begin{enumerate}
\item a literature review of central papers on subject themes found
  with prepared search statements
\item an action research using a laboratory setup
\item a quantitative research process using QuERIES methodology
\end{enumerate}

Literature review will be addressed in the next section. As this
thesis' main research methodology is quantitative, the gathered data
points will be addressed as variables that are compared with
mathematical methods. Quantitative methods are broken down in chapter
\ref{securitystandpoints} and \ref{analysis}. The central methodology
is derived from QuERIES, and the information security aspect is
investigated with CIA-triad. In chapter \ref{analysis} section
\ref{whyqueries} QuERIES is compared with other metrics which are
explained in subsection \ref{resquest}.

Quantitative methodologies are oftentimes used in conjunction with
qualitative methodologies, both approaches having their strengths and
weaknesses. One drawback of using qualitative methods in security
framework is their inherent subjectivity. For example the Delphi
technique, where a set of opinions is gathered and compared from a
working group provides subjective substance for a study instead of
objective perspectives \cite{wang2005information}.

The study design in this thesis is \textit{state based}, which refers
to the fact that the research methods focus on different state
transitions, e.g how probable it's for an intruder to gain from
partial leverage to a full control of a system. Qualitative research
wouldn't alone satisfy the requirements, as investigating different
state transitions without quantitative methodologies would be absurd
\cite{ramos2017model}.

\subsection{Literature review}\label{litrev}

This thesis has bibliography from x sources, most of it gathered with
a carefully prepared search statement. Other sources include manuals,
material for research methods and other relevant material. The search
statement's results, presented in next subsection, provide good base
for action research and analysis.

The literature focuses on three main concepts: embedded systems
security with and without declarative components, imperative systems
and NixOS as a solution. The main goal is to find literature that
combines these concepts to gain platform for comparing different
approaches to support the action research.

Central literature revolves around Nix and multiple texts by Eelco
Dostra are cited for illustrating the nuances of a Nix
ecosystem. Other declarative approaches are discussed, by Endres
et. al and Van der Burg \cite{van2010declarative,
  endres2017declarative}. These approaches also contains comparison to
imperative systems, which is the central approach in chapter
\ref{imperative}. Combining cyber security with declarative approaches
were discussed by Specht et. al and Kandoi and Artke
\cite{specht2007analysis, kandoi2021operating}.

Discussion from Ravi et. al and Fysarakis et. al on embedded security is discussed by
\cite{ravi2004security, fysarakis2014embedded}. The concepts, however
are generally too broad for this thesis' scope, so only the most
fitting approaches were selected for use.

\subsubsection{Search statement} \label{searchstatement}

The main search statement for this thesis is: "embedded linux" OR
''declarative'' AND (linux OR *nix) OR deployment OR ''system update''
OR (compare* AND declarative AND imperative AND system*) OR security.

The search statement was prepared to provide as relevant results as
possible for this thesis. The main goal was to include the hypernyms
''embedded linux'', ''linux'' with other terms separated using the
"OR" operator. The subterm (compare* AND declarative AND imperative
AND system*) was chosen to broaden the search to include articles
which compare declarative and interactive systems.

As security is a central theme in this thesis, the term "security" was
included. Search was done on Google Scholar, and other useful material
was handpicked, such as NixOS manuals and wiki pages. Systems security
and cyber security material is also included in the bibliography using
search statement systems security OR cyber security. Separate search
''cia-triad'' and ''partial observable Markov chain'' AND
''cybersecurity'' were used to provide tangible meters for
measuring cyber security. To further back up the research for
comparing different metrics, term ''cyber security metric
methodology'' was searched.

For searching specific material about embedded systems, the search
statement ''embedded AND security'' was used. As the need for embedded
toolchains was needed, statement ''yocto AND buildroot'' was
searched. All searches were done on Google Scholar platform.

\subsection{Research questions} \label{resquest}

The research questions for this thesis are:

\begin{enumerate}
\item How can a declarative system be used to improve the basic
  security needs of an embedded system used for displaying public
  media measurably?
\item What are the advantages and/or disadvantages of such system from
  system administrator standpoint?
\item How can a declarative system be updated from different Linux
  distribution securely and seamlessly?
\end{enumerate}

Research question 1 is perhaps the most important and it traverses
through themes of the whole thesis. The hypothesis is that traditional
imperative embedded device fleets have problems that can be solved
with the use of modern declarative systems. First, we aim to gain
information from a specific scenario, presented in chapter
\ref{architecture}, then in chapter \ref{analysis} the gained
information is analyzed and generalized as suitably as possible.

Research question 2 brings up the human element; how can a system
administrator use a new palette of features adequately to provide more
secure system and research question 3 handles a situation where
existing system should be replaced with a NixOS system. How this
could be done securely without risks and preferably easily with
existing or new tooling is answered in chapter \ref{architecture}
section \ref{instnewdevices}.

\subsection{Data collection and analysis}

Data collection is done with simulated red–blue team setup, where
either team has a time frame where they must conduct a series of
tasks. These tasks are formalized as partially observable Markov chain
parameters, and analysed with QuERIES methodology. This methodology is
used to gain knowledge and make the system more reliant and better
with multiple iterations. Chapter \ref{analysis} answers research
question 1 and 2 and provides analysis for the reference
system. Research question 3 is answered in chapter \ref{architecture}
section \ref{instnewdevices}.
