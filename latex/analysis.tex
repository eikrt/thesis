
\chapter{Analysis} \label{analysis}

As mentioned in the previous chapter section \ref{queries}, the
QuERIES inspired methodology is used in this chapters analysis. Tables
\ref{pomdbtable} and \ref{probtable} show relevant information about
the calculations. In table \ref{pomdbtable} is listed all the possible
states (S), defender actions (D), attacker actions (A) and
observations (O). Then, every transition from state to another state
was calculated as a probability, shown in table \ref{probtable}. The
test counts are seen from the table cells, with their corresponding
probability. Note that every row adds up to 1.0.

\section{Modeling the problem and quantifying the models} \label{modprob}

The example project of this thesis is an image showing system that
could be used e.g for advertisement, public transport timetables or
practically anywhere where static media should be presented. Our use
case focuses on displaying arbitrary media in a public location. The
results, if an an intruder should gain unauthorised access, would be
anywhere from displaying improper imagery, to succeeding in displaying
propaganda or other unwanted content. Unauthorised access could have a
negative economic effect for the service provider, as every
organisation displaying media want to remain credible among users.

The attack surface of the example project focuses on physical access
and vulnerabilities in remote connections. With MQTT-messaging, SSH
and display protocols, internal and external messaging takes place.

If a unauthorised access would happen, the results would probably
affect a part of the society, as the arbitrary content could gain
media and social media attention. This public humiliation would
definitely affect the credibility of the service provider, as well as
the customer. Possible propaganda could affect society by spreading
false information, or possibly bringing up societal issues via
activism. Either way, this would be unwanted from the perspective of
the service provider, customer and users.

As mentioned in chapter \ref{securitystandpoints} section
\ref{measuringsecurity} subsection \ref{queriesasmethodology}, the
value of intellectual property is defined as α, for which other
parameters refer to as fractions of it. If this was applied to a real
setting, the intellectual property would be calculated appropriately
for the scenario.

\section{Modeling the possible attacks}

In this section, the possible attacks are modeled using a attack
chart, depicted as a POMDP, a modeling tool presented earlier. In the
original paper where QuERIES is presented, an important parameter is
the time to reverse-engineer the system without prior information
about the protection scheme \cite{carin2008cybersecurity}. Our
approach is different; the attacks are considered as successful, if
they gain further leverage in the attack graph, e.g transitioning
state ''idle'' to ''partial loss of system''. This is to maintain
cohesion in the study, as we don't need to define what it means to
''reverse engineer'' the system. In this case, information of the
system is publicly available as an open source project, thus the
information of the system being available also to the attacker.

In table \ref{pomdbtable}, the first column describes states the
system can be in. The second and the third column state defensive
actions, and observations of the system. The fourth column contains
template of the attacks that could be conducted. After defining the
starting layout, probabilities are calculated based on gathered
empirical evidence.

The research setup is a simulated targeted red–blue scenario. For
calculating the reward values R script with library ''pomdp'' was
used.
\begin{table}
\centering
\begin{tabular}{ |c|c|c|c|c|c|c|c|c|c| }
\hline s&c_t&s_0&s_1&s_2&s_3&s_4&s_5&s_6&s_7
\\ &&c,p&c,p&c,p&c,p&c,p&c,p&c,p&c,p \\ \hline s_0&&&&&&&&& \\ \hline
s_1&&&&&&&&& \\ \hline s_2&&&&&&&&& \\ \hline s_3&&&&&&&&& \\ \hline
s_4&&&&&&&&& \\ \hline s_5&&&&&&&&& \\ \hline s_6&&&&&&&&& \\ \hline
s_7&&&&&&&&& \\ \hline
\end{tabular}
\caption{Probabilities of each state transition occurring in a 7 state
  Markov chain, where s represents state, c count and p probability of
  the transition succeeding. \(C_t\) represents total count. Given
  action is ''Monitor state''.}
\label{probtable}
\end{table}
\begin{table}
\centering
\begin{tabular}{ |c|c|c|c|c|c|c|c|c|c| }
\hline o&c_t&s_0&s_1&s_2&s_3&s_4&s_5&s_6&s_7
\\ &&c,p&c,p&c,p&c,p&c,p&c,p&c,p&c,p \\ \hline O_0&&&&&&&&& \\ \hline
O_1&&&&&&&&& \\ \hline O_2&&&&&&&&& \\ \hline O_3&&&&&&&&& \\ \hline
O_4&&&&&&&&& \\ \hline

\end{tabular}
\caption{Probabilities of observation across states.}
\label{probtable}
\end{table}
\begin{landscape}
\begin{table}
\centering
\begin{tabular}{ |c|c|c|c| }
 \hline & & & & S0-S7 & D0-D4 & O0-O4 & A0-A4 \\ & & & & \hline \hline
 & & & & Idle & Monitor system & Normal operation & Intercept MQTT
 messaging \\ & & & & \hline & & & & Receive media through MQTT &
 Patch system & Detected suspicious activity & Compromise Github
 repository \\ & & & & \hline & & & & Set up SystemD services &
 Shutdown system & Detected system error & Gain physical access to
 device \\ & & & & \hline & & & & Start Weston & Isolate device &
 Detected unusual media display & Exploit vulnerabilities in display
 \\ & & & & \hline & & & & Display media & & & Exploit vulnerabilities
 in SSH connections \\ & & & & \hline & & & & Error state & \ & \ &
 \\ & & & & \hline & & & & Partial loss of system & & & \\ & & & &
 \hline & & & & Complete loss of system & & & \\ & & & & \hline

\end{tabular}
\caption{Different states, defensive measures, observations and attack
  measures for the system.}
\label{pomdbtable}
\end{table}
\end{landscape}

A reward function was calculated using the mentioned R program, and
the output was x. A weight of 1 was used for positive results, and
-100, if something was to be compromised. This weight distribution is
due to the fact that even if blue team succeeds most of time, the
results of failure are much worse than a succeeding result from the
blue team \cite{carin2008cybersecurity}. The discount constant is used
in calculations as shown in \ref{pomdpappenix}, where it influences
the priority of immediate versus future rewards
\cite{mcabeeMarkov}. Our case signifies the importance of both, so
value of 0.75 was used.

\section{Using the results}

The results are investigated through the probabilities, which are
represented in the table \ref{probtable}. The reward score is taken in
to account on how successful/unsuccessful the setup is from a security
perspective. 

As stated by Hughes and Cybenko, using the results means evaluating
the gained results to decide if proposed protections are adequate for
our means \cite{hughes2013quantitative}. The QuERIES model was
iterated 4 times, and the results were placed in the following table. 

\begin{table}[h!]
\centering
\begin{tabular}{|c|c|c|}
  \hline
  Iteration & Reward function & Proposed protections \\
  \hline
  1 & -200 & Disabling USB ports \\
  \hline
  2 & -100 & Doing sum stuff \\
  \hline
  3 & 0 & Moar stuff \\
  \hline
  4 & 100 & No protections \\
  \hline
\end{tabular}
\caption{Protections applied in each iteration}
\label{iterationtable}
\end{table}

As seen, the reward function is growing as the proposed protections
are applied in each iteration. In the following figure we can see that
the time to breach peaks at time x. This highlights the fact that
.... It's similiar to results seen in...

\begin{figure}[t!]
\centerline{\includesvg[width=1.0\columnwidth]{../graph/barchart.svg}}
\caption{Probabilities to breach into the system.}
\label{timetobreach}
\end{figure}

\subsection{Red team implications}

The time distribution of breaches are depicted in the graph
\ref{timetobreach}. It's apparent, that ... The chart is used to
decide, when is the optimal time to stop the attacks from the attacker
perspective. Identical approach is taken with blue team in the
next subsection. Following the methodology in referred papers by Hughes and Cybenko,
the optimal time to stop the attacks is seen in the following graph
using open and closed loop strategies.

\begin{figure}[t!]
\centerline{\includesvg[width=1.0\columnwidth]{../graph/barchart.svg}}
\caption{The most optimal time to stop the attack.}
\label{openandclosed}
\end{figure}

%The result, -30 is negative and that proves that the system has definitive security problems. From the table \ref{probtable} it can be seen that multiple attack vectors succeed in penetrating through the system. The worst probabilities occur when physical access is gained, due to the test device having exposed USB-ports. This weakness could be mitigated, if the production device would have closed USB-ports and a sufficient casing against physical breakthrough. On the other hand, replacing the whole device with a malicious one can't be prevented without the use of external security measures.

\subsection{Blue team implications}

\begin{figure}[t!]
\centerline{\includesvg[width=1.0\columnwidth]{../graph/barchart.svg}}
\caption{The most optimal time to stop the attack.}
\label{openandclosed}
\end{figure}

%The blue team tests proved, that the system is very reliable in every state. For example, monitoring the system always worked, and logging could be obtained. Recovering from simulated errors was simple, as the system is very straight forward. A more complex system would perhaps have more issues.

%It's worth noticing that even partial loss of system can be catastrophic, as display servers such as Weston don't need root privileges to operate. This is due security reasons, but comes with the feature that anyone with the user access might display arbitrary content. This is an easy fix, however, because the use of Weston can be denied for all, except for the SystemD user.

% A predefined topology could be used to mitigate device attacks based on device replacing.\cite{fysarakis2014embedded}
%% \begin{table}
%% \centering
%% \begin{tabular}{ |c|c|c|c|c|c|c|c|c| }
%%   \hline
%%   &&&&&&&&& \\
%%   \hline
%%   &&&&&&&&& \\
%%   \hline
%%   &&&&&&&&& \\
%%   \hline
%%   &&&&&&&&& \\
%%   \hline
%%   &&&&&&&&& \\
%%   \hline
%%   &&&&&&&&& \\
%%   \hline
%%   &&&&&&&&& \\
%%   \hline
%%   &&&&&&&&& \\
%% \hline
%% \end{tabular}
%% \caption{Different outcomes of various results gained via QuERIES}
%% \label{probtable}
%% \end{table}
%\cite{papazov2019cybersecurity}
