\chapter{Applying QuERIES} \label{analysis}

As mentioned in the previous chapter section \ref{queries}, the
QuERIES inspired methodology is used in this chapters analysis. In
table \ref{pomdbtable} is listed all the possible states (S), defender
actions (D), attacker actions (A) and observations (O). Then, every
transition from state to another state is calculated as a probability.

QuERIES as a model is described as multidisciplinary, as it takes
inspiration from traditional red-blue team approaches, mathematical
models to economic models \cite{hughes2013quantitative}. In other
literature, particularly by Bojanc and Jerman-Blažič, it's argued that
modeling cybersecurity with economic models can provide substance for
minimizing risks organization-wide \cite{jerman2008economic}. In this
chapter, we will apply the model and discuss the implications in both
security and economic standpoints. The main focus however is on how
using QuERIES, we can improve the security of the architecture.


\section{Modeling the problem and quantifying the models} \label{modprob}

The example project of this thesis is an image showing architecture that
could be used e.g for advertisement, public transport timetables or
practically anywhere where static media should be presented. Our use
case focuses on displaying arbitrary media in a public location. The
results, if an an intruder should gain unauthorised access, would be
anywhere from displaying explicit imagery to succeeding in displaying
propaganda or other unwanted content. Unauthorised access would have a
negative economic effect for the service provider, as every
organisation displaying media want to remain credible among
stakeholders.

The attack surface of the example project focuses on physical access
and vulnerabilities in remote connections. With MQTT-messaging, SSH
and display protocols, internal and external messaging takes place.

As mentioned in chapter \ref{securitystandpoints} section
\ref{measuringsecurity} subsection \ref{queriesasmethodology}, the
value of intellectual property is defined as $\alpha$, for which other
parameters refer to as fractions of it. If this was applied to a real
setting, the value of intellectual property would be calculated
appropriately for the scenario.

\section{Modeling the possible attacks}

In this section, the possible attacks are modeled using an attack
chart, depicted as a POMDP, which is a modeling tool presented
earlier. In the original paper where QuERIES is presented, an
important parameter is the time to reverse-engineer the system without
prior information about the protection scheme
\cite{carin2008cybersecurity}. Our approach is different; the attacks
are considered as successful, if they gain further leverage in the
attack graph, e.g transitioning state ''idle'' to ''partial loss of
system''. This is to maintain cohesion in the study, as we don't need
to define what it means to ''reverse engineer'' the system. In this
case, information of the system is publicly available as an open
source project, thus the information of the system being available
also to the attacker.

In table \ref{pomdbtable}, the first column describes states the
system can be in. The second and the third column state defensive
actions, and observations of the system. The fourth column contains
template of the attacks that could be conducted. After forming the
attack graph depicted in the table, the attacker tries to breach the
system, leveraging through the A0-A4 column. The probabilities are
then calculated with the model presented in chapter
\ref{securitystandpoints} section \ref{quantitativemetrics} subsection
\ref{queries}.

The algorithm produces a reward value for each QuERIES iteration,
which is used to improve the setup. For calculating reward values, R
script with library ''pomdp'' was used.

\begin{landscape}
\begin{table}
\centering
\begin{tabular}{ |c|c|c|c| }
 \hline & & & & S0-S7 & D0-D4 & O0-O4 & A0-A4 \\ & & & & \hline \hline
 & & & & Idle & Monitor system & Normal operation & Intercept MQTT
 messaging \\ & & & & \hline & & & & Receive media through MQTT &
 Isolate system & Detected suspicious activity & Compromise Github
 repository \\ & & & & \hline & & & & Set up SystemD services &
 Shutdown system & Detected unauthorised access & Gain physical access
 to device \\ & & & & \hline & & & & Start Weston & Isolate device &
 Detected unusual media display & Exploit vulnerabilities in display
 \\ & & & & \hline & & & & Display media & & & Exploit vulnerabilities
 in SSH connections \\ & & & & \hline & & & & Partial loss of system &
 \ & \ & \\ & & & & \hline & & & & Complete loss of system & & & \\ &
 & & & \hline & & & & & & & \\ & & & & \hline

\end{tabular}
\caption{Different states, defensive measures, observations and attack
  measures for the system.}
\label{pomdbtable}
\end{table}
\end{landscape}

A weight of 1 was used for positive results, and -100, if something
was to be compromised. This weight distribution is due to the fact
that even if blue team succeeds most of time, the results of failure
are much worse than a succeeding result from the blue team. The
discount constant is used in calculations as shown in appendix 
\ref{pomdpappenix}, where it influences the priority of immediate
versus future rewards \cite{mcabeeMarkov}. Our case signifies the
importance of both, so value of 0.75 was used. Note that the maximum
reward value is 4, and the minimum is -400.

\section{Using the results} \label{usingtheresults}

The results are investigated through the probabilities, which are
represented in the table. The reward score is taken in account on how
successful/unsuccessful the setup is from a security perspective. The
score itself is rather abstract; it's main function is to demonstrate
the \textbf{overall security} of the architecture. In short, negative
numbers imply there are flaws in security, and positive numbers mean
that the system is more secure
\cite{mcabeeMarkov}.

As stated by Hughes and Cybenko, using the results means evaluating
the gained results to decide if proposed protections are adequate for
our means \cite{hughes2013quantitative}. The QuERIES model was
iterated 2 times, and the results were placed in the following table.

\begin{table}[h!]
\centering
\begin{tabular}{|c|c|c|}
  \hline Iteration & Reward function & Proposed protections \\ \hline
  1 & -25 & Multiple \\ \hline 2 & 4 & None \\ \hline
\end{tabular}
\caption{Protections applied in each iteration}
\label{iterationtable}
\end{table}

The reward function in the first iteration is calculated with the
script. The second value is calculated
with as the maximum accumulated points; the highest value is 4 points
without any discount, as the red team failed to provide any results
for the second iteration. This is partially due to time limit, as in
the test set up there was only one attacker with very limited time. I
argue that with more time, it would have been possible for the
attacker to breach.

As seen, the reward function is growing as the proposed protections
are applied in each iteration. In the following figure
\ref{timetobreach} we can see that the time to breach peaks in the
first hour.

\begin{figure}[t!]
\centerline{\includesvg[width=1.0\columnwidth]{../graph/barchart.svg}}
\caption{Probabilities to breach into the system.}
\label{timetobreach}
\end{figure}

The time distribution of breaches are depicted in the graph
\ref{timetobreach}. The chart has similar shape as the chart in
\cite{carin2008cybersecurity}, meaning that attacks in their initial
phase succeed more frequently at least in these two cases. The following chart is used to decide, when is the
optimal time to stop the attacks from the attacker perspective. We use
this also to reflect the cost estimate of the blue team. Following the
methodology in referred papers by Hughes and Cybenko, the optimal time
to stop the attacks is seen in the following graph using open and
closed loop strategies \cite{hughes2013quantitative,
  carin2008cybersecurity}.

This economic model can be used to generalize risk in this kind of
situations. In the following graph \ref{openandclosed} we can see that
the cost for the attacker peak at about 6 hours using closed algorithm, and
at 3 hours using open algorithm. Open loop algorithm means that the cost
estimate never evaluates feedback from a system
\cite{bars2006theory}. Closed loop behaves in an opposite manner; it
constantly modifies its behaviour based on the feedback it gets \cite{bars2006theory}.

We can see that with closed loop algorithm, the attacker can gain more
results with worse cost tradeoff, but with open loop, it gets
immediate results with cheap cost. This could be important for the
holder of the intellectual property, as it must protect itself against
attackers using both algorithms. In our case, where the potential
threats are e.g hacktivist groups, the cost estimate doesn't matter as
much. Therefore, it's important to be aware that an attacker which
doesn't have economic interests, could have practically unlimited time
to try to breach the system.

\begin{figure}[t!]
\centerline{\includesvg[width=1.0\columnwidth]{../graph/cost_graph.svg}}
\caption{Cost benefits and reductions. The black dots indicate the
  most optimal times to stop the attack.}
\label{openandclosed}
\end{figure}

\subsection{Improving the system using Nix}

The following measures were taken in the first iteration: disable USB
ports, change MQTT password to token based authentication and change
SSH to key based authentication. As we can see, the QuERIES output
provided concrete results on what are the weak points of the system.

As an example, the exposed USB-ports of the system were found to be a
problem, due to the possibility of infecting the system with direct
control to the system. This can be mitigated easily by modifying the
client devices configuration.nix, with the following changes:

\begin{figure}[H]
\begin{lstlisting} 
{ boot.kernelPackages = pkgs.linuxPackages_latest; boot.kernelParams =
  [ "nousb" # Disables USB at the kernel level ]; boot.kernelModules =
  [ ]; boot.extraModulePackages = [ ]; services.udev.extraRules = ''
  SUBSYSTEM=="usb", ACTION=="add", OPTIONS+="ignore_device" ''; }
\end{lstlisting}
\label{kernelsnippet}
\end{figure}

The configuration could be applied to any number of clients, proving
that Nix could be used to rapidly address arising security
issues. This supports the argument presented in chapter
\ref{imperative}, that through updatability the security could be
improved. The configuration is only an example to mitigate hardware
access attack vectors, as there are many more ways for the attacker to
leverage the access, e.g by replacing the whole device, depending on
the resources of the attacker. This highlight the scalability of Nix
systems; it doesn't matter if we have one or thousands of devices,
updating them is equally simple. A huge contrast between imperative
and declarative systems is found, as imperative systems would need
linear administration time in relation with the number of devices.

\subsection{Issues with QuERIES} \label{issues}

One issue found using QuERIES with POMDP was that applying the reward
functions is in our case rather arbitrary. We assumed, that for
positive results, the reward function is defined as 1, and for the
negative outcomes -100. This is due to the negative results deemed as
catastrophic and the positive results being slightly positive. The
choice for both parameters could have been any integer, but the issue
is that it demands ''gut-feeling'' of the author to select the
appropriate parameters. This could be mitigated with some study on
selecting the paramers, but it's definitely a challenge to provide
good foundation on such abstract numbers.

One remark using POMDP is that it produces very generalized
output. This is why we use methodologies such as QuERIES to improve
the system, POMDP being just one component. Other benefit from using
POMDP is that it provides us concise attack graph, and I argue that
using the POMDP calculations may even be redundant. In POMDP, negative
rewards signify that the system has issues, and positive meaning that
the system is more secure \cite{mcabeeMarkov}. More ergonomic approach
would be calculating rewards without positive outcomes, due to them
possibly shadowing the most critical issues. Developing a pessimistic
model that highlights explicitly the weak points of a system would be
a topic for further research.

\section{Generalizing the results}

The results of this chapter could be generalized to a more complex
setting, using a similar methodology. The POMDP model scales well, if
more parameters would be supplied. This study proves that while
measuring cybersecurity is difficult, remaining in tight constraints
we can get adequate estimates for applying economic models for the red
team, further projecting the risk vs. reward.

Carin et. al \cite{carin2008cybersecurity} argue that QuERIES can be
applied in both public and private sectors to help to improve the
security of both software and hardware. I argue that a hand-tailored
application of QuERIES can be used as a powerful model that is
ergonomic to use in the right hands. However, the use of complex
applied mathematic models require proficency in both cybersecurity and
mathematics, which can be hard to achieve in a real organizational
setting. This is why I propose in the next chapter that a simpler,
pessimistic and more concise model would be easier to reach for an
organization.
