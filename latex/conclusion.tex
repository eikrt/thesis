\chapter{Conclusion} \label{conclusion}

Nix can be a useful base for issuing a purely functional and
declarative systems while providing an advanced rollback
mechanism. Configuring it, depending on the system administrator, may
be gruesome. However, it can dodge the usual pitfalls of more popular
systems. While the usage of Nix language demands proficiency, it is a
solution for distributing, updating and administrating more secure
systems. NixOS handles features such as Wayland and audio driver setup
effortlessly. Ultimately, NixOS is very predictable and the statement
"if it works on one machine, it works on another" proves to be true.

Problems presented in this thesis could be solved with many different solutions,
but the Nix approach is rather unique as it makes reproducible systems straight forward. As the system is defined in the
configuration.nix file, there is little that could go wrong even if 
underlying hardware varies. The same does not apply to the mentioned
Debian distribution.

Measuring and improving the architectures solution proved to be
challenging but fruitful. With multiple iterations used with QuERIES,
the architecture developed from insecure to more secure. This is a
testament for applying a certain methodology for a
specific task successfully.

As far as security implications are concerned, Nix is a good tool for
asserting compliance. In purely functional environment, anything that
needs compliance can be checked from the configuration files. As disk
encryption is trivial and read-only, Nix store provides increased
integrity. A system administrator who is proficient with Nix, can
assess the states of the systems' very quickly. For example, viewing
the open ports of each device can be answered quickly and in a precise
manner by simply viewing the configuration files.

Popular embedded Linux distribution creating tools,
Buildroot and especially Yocto have also steep learning curves as well. Furthermore, they
lack the ease of updatability provided by Nix, which is a
significant problem regarding security.

The research proved, that the purely functional Nix system is an
easily updatable and upgradable solution for maintaining a secure
system.  While NixOS is a niche Linux distribution, and probably
remains as such, the concepts however will likely live on in the
future. A purely functional declarative system with atomic rollbacks
is something to be iterated in the future with further user
availability in mind.
