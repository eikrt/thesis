\chapter{Further research} \label{further}

One big limitation of Nix is the fact that it mainly supports only
x86\_64, i686 and arm64 platforms \cite{nixosNixOSManual}. This is not
generally enough, as many different architectures are used in
embedded, thus limiting the use-cases of
NixOS \cite{fysarakis2014embedded}.

A declarative approach could be used with other processor
architectures. This would require work on a declarative package
manager, which would be used in conjunction with base systems created
with Buildroot or Yocto.

As far as security is concerned, the main methodology of this thesis'
research, QuERIES, proved to be an aqequate tool which helped to
improve the security of the reference architecture. However, some
issues were raised in chapter \ref{analysis} section
\ref{usingtheresults} subsection \ref{issues}. These issues could be
addressed in developing a new methodology inspired by QuERIES,
combining the acquisition of POMDP parameters methodically without estimating them
in conjunction more reachable calculation approach.

\section{Further research questions}

Questions for further research include:
\begin{enumerate}
  \item{What kind of new methodology would be better, reflecting
    the found issues in QuERIES for similar use-case as in this
    thesis?}
  \item What set of tools would be optimal in creating a secure true
    cross-platform declarative package manager 
    
\end{enumerate}

Developing a new Linux distribution would be a very big task, so
implementing support for different architectures for Nix or Guix for
embedded use would be the focus of further work. These approaches
would hold the same arguments as in this thesis to improve the
security of embedded devices.

