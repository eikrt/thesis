\keywords{declarative, nix, nixos, pomdp, security, queries}
\keywordsen{declarative, nix, nixos, pomdp, security, queries}
\begin{abstract}

Embedded devices are an integral part of our daily lives; household
machines, automobiles, and thermal sensors make use of embedded
devices. They are subject to the global, developing worlds' security
problems. This thesis focuses on those found on public information
screens. Embedded devices are particularly vulnerable to security
problems as they face issues in receiving constant, reliable
updates. This thesis' focal point is maintaining, updating, and
upgrading embedded devices. A proposed
architecture of a public media screen system is provided with example
program snippets to cover most common security issues found in similar
setups. The architecture and its content are evaluated through
the QuERIES methodology. The central theme of this thesis is NixOS,
which is a Linux distribution that forms itself from a set of
configuration files, supporting features like atomic rollbacks and
reliable dependency handling. The most definitive academic sources in this
particular subject are used extensively, as well as papers regarding
both embedded security and measuring security.

A quantitative research methodology, QuERIES is used to measure the
security of a novel architecture using NixOS. QuERIES contains a
number of steps that evaluate the security of a system. The steps are
iterated two times, each iteration providing a partially
observable Markov decision process (POMDP) output, which is used as a
benchmark of the overall security of the reference architecture.

The result of this thesis is that with the use of QuERIES, the overall
security of the architecture can be improved methodically with the use
of POMDP as a defined attack graph. An economic model of cost
estimation to the attacker is gained via the red-blue team setup,
which is then used as a tool for revealing the weak spots of the
architecture from a chronological standpoint. The output of QuERIES
can be generalized with tight constraints; as QuERIES provided
tangible improvements in small scale, it could serve well a more
complex setting. This is due to the nature of QuERIES in tandem with
POMDP being able to handle a number of parameters, which is essential
in a larger setting.

Ideas for further work are presented for studying the possibilities of
purely functional and declarative solutions in the embedded
field. The issues in QuERIES are also highlighted, and the development of more
accessible tools for measuring cybersecurity is discussed.

\end{abstract}
