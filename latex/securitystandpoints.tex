\chapter{Security standpoints} \label{securitystandpoints}

Security is inherently challenging to measure adequately, due to its
complex and chaotic nature. Qualitative analysis may also result in
subjective output. As such, no unambiguous standard of measuring
security can be provided \cite{wang2005information}. To overcome these
challenges, several metric systems are compared, and the one that
provides the most precise answers specifically for this thesis' cause
is chosen.

In this chapter, selection process for meters, which in part are used
for gaining quantified data, is presented.  When finally quantitative
metrics are gained with the help of two paradigms, GQM (goal, question, metric) and
specifically its superset SMART, a red–blue team layout is erected
using QuERIES methodology. SMART is used in conjunction with the
literature review to reveal the most suitable methodology for this
thesis. SMART is opened more in the section \ref{choosingsecmet}.

Using an exact metric system is important, as this this thesis'
emphasis is on measuring the security of a declarative approach. The
presented architecture in the previous chapter works as a example on
how a declarative system can be improved by reconfiguration, utilizing
a red-blue team setup.

\section{A brief history of security metrics}

The history of security metrics begin from Trusted Computer System
Evaluation Criteria (TCSEC), also known as the Orange Book from 1983
which popularized many terms still in use today, such as
identification, authentication and
authorization. \cite{bayuk2013measuring}

When US National Bureau of Standards (NBS), the organization later
formed as the National Institute of Standards and Technology (NIST)
tried to standardize security, in the 1990s it became evident that a
system should adhere too strongly to the definitions of the Orange
Book and the follow up project, Common Criteria to be used
accordingly. Later, multiple standards became popularized, e.g System
Security Engineering Capability Maturing Model (SSE-CMM), which can be
used as a sort of checklist from system design from the ground
up. \cite{bayuk2013measuring}

Later it was observed that systems design is only a part of a
successful security strategy, and operational practices played a
bigger role than expected, something to be addressed in 1995 NIST
Computer Security Handbook, which has evolved to provide ground to
combat modern issues. \cite{bayuk2013measuring}

As metrics can be used as a tool for decision making, the strategical
approach of the mentioned publishes is important. It's noteworthy that
the strategies (the Orange Book, SSE-CMM, etc.) begin to measure
security by compliance to defined ratings. Later in 2000s more
mathematical approaches were taken, one which is delved deeper in
section \ref{whyqueries}. \cite{bayuk2013measuring}

\section{Choosing security metrics} \label{choosingsecmet}

Security is something that is challenging to measure due to it's
complex nature. A GQM (Goal, question, metric) paradigm helps to
choose appropriate metrics: first there must be a set goal to a
organisation, then a formulated question for each goal. These answers
are then reflected to gain the desired metric. This strategical
approach is perhaps too broad for this thesis' scope, but aligns well
with an usual organisational strategy. \cite{papazov2019cybersecurity}

A more appropriate tool for this task would be SMART – a set of inputs
to evaluate meter systems' suitability. These inputs describe how
specific, measurable, attainable, relevant and timely the methodology
is \cite{payne2006guide}.

In cyber security, being specific is very important and a common issue
with security meters is that they either cover too many topics and are
without precise definitions, or they are too specific to be
generalized to broader scope of situations
\cite{wang2005information}. This thesis' results are important to be
measurable, as the research orbits around system states, and the
research aims to measure with what outputs do the state transitions
resolve to.

To be attainable is relevant to this context for the reason that a
thesis has different scope than a big organization, and the
proposed setup has to go through a check on the metrics' goals
achievabilities. Relevance has to do with risk assessment: how important
it's to measure something related to it's value. Risk assessment is
gone through thoroughly in \ref{analysis} section
\ref{modprob}. Time-bounding signifies the importance of time as a
meter; a system that can be penetrated in a minute can definitely be
seen as weaker than a system that takes years to be compromised.

\section{Measuring security} \label{measuringsecurity}

In this section, different methodologies and perspectives gained
through literature review for cybersecurity are discussed, and
potential methodologies are compared to gain the most adequate metric
system for usage through SMART process \cite{payne2006guide}.

Security metrics can be divided to address four separate themes:
\textbf{System vulnerabilities}; measuring vulnerabilities can be
applied to user, interface-induced, password, and software
vulnerabilities. Users are always susceptible to e.g phishing attacks
or malware infection, where a user of an arbitrary system is the
definitive attack vector. Interface-induced vulnerabilities refer to
attack vectors related to open ports and
endpoints. \cite{pendleton2016survey}

Password vulnerabilities refer to situations where password can be
computationally cracked. This is relatively simple to measure, as it
can be estimated how much time it takes to crack a password, or with
the use of statistical password guessability. Software vulnerability
on the other hand is a very usual way for a breach to take
place. This kind of vulnerabilities can be measured, thus also
estimated with the help of exploitations in the past. The
essential metric is the time to patch a software vulnerability. \cite{pendleton2016survey}

\textbf{Defense} measures can be applied to strength of reactive,
preventive proactive and overall defenses. Reactive measures include
blacklisting, a lightweight mechanism to prevent e.g a botnet to harm
the protected system by blacklisting IP-addresses related to the
botnet. For measuring defence, the reaction time is essential, and
most importantly, the gained meter to measure preventive
defense. Blacklisting can also be used as preventive and proactive
measure, as a pre-filled blacklist can be used with desired
parameters. \cite{pendleton2016survey, ramos2017model}

Overall defenses can be measured with the combination of all defensive
measures and with the use of penetration testing in a red–blue team
setup. Penetration testing aims to gain a result, also known as
penetration resistance, which is a meter, indicating cost or time that
the red team must spend in case of a successful system
compromisal. \cite{pendleton2016survey, ramos2017model}

\textbf{Threats}: zero-day vulnerabilities can be measured from two
perspectives: lifetime of zero-day vulnerability and the number of
nodes that are compromised as a result. Malware spreading can be
traced with the parameter infection rate, which is defined as infected
node per a time unit. Attack evasion is measured using either
obfuscation prevalence metric, or structural complexity metric which
provide information on obfuscating gained samples e.g by encrypting,
or the target system's complexity measured by
runtime. \cite{pendleton2016survey, ramos2017model}

\textbf{Situations}, which can relate to security state, security
incidents and security investments. Security state has multiple
parameters, including incident rate and blocking rate. Security
investments on the other hand measure the budget percentage funneled
towards security, and the return of such
investment. \cite{pendleton2016survey}

\section{Methodologies in comparison} \label{whyqueries}

Today, there exists cybersecurity metrics based on quantified
mathematical models, which are prevalent for this thesis. Three
different methodologies are discussed, and one is picked for measuring
the security of this thesis' architecture implementation. All the following metric
systems are \textbf{measurable}, but some fit better especially
according to \textbf{time-related} and \textbf{relevance} axes.

The literature review provided three central methodologies from different perspectives. Complex mathematical models
presented by Alshammari et al. are too broad
\cite{alshammari2009security}. This thesis' scope is limited, and this
methodology would fit better a wider cyber security setting.

A methodology based on object-oriented thinking, followed by
UML-graphs is adequate in many contexts, but as stated in the paper
this measurement the methodology is used to compare similar
alternative designs. \cite{alshammari2009security}. As it has been
stated, this thesis focus isn't comparative, as all the critical
comparison has already done in chapter \ref{imperative}.

Hidden Markov models presented by Wang et al. are close what is the
end goal of the research is in this thesis
\cite{wang2010framework}. The time-related aspect would be
satisfactory, as the hidden Markov model handles a time
parameter. The problem, however is regarding the generality of the
methodology, and something more specific would be a better
fit for this thesis. We however use a similar approach as what Wang et
al. use for calculating the POMDP parameters.

The last and the most fitting methodology would be one presented in
papers by Carin et al and Hughes, Jeff and Cybenko
\cite{carin2008cybersecurity, hughes2013quantitative}. The
QuERIES methodology is delved deeper in section \ref{queries}, and
its time-related, relevance and specific
axes are a near-perfect match for our goals as the model itself is
relatively simple and provides shifting probabilities from states,
which serves this thesis' study design well.

\section{Quantitative metrics} \label{quantitativemetrics}

Selecting carefully a metrics system includes asserting our goals and
questions. Our goal is to discover this thesis' architecture proposals
tenacity in a simulated setting. The main goals reside in this thesis'
two research questions:

\begin{itemize}
\item How can a declarative system be used to measurably improve the basic
  security needs of an embedded system used for displaying public
  media?
\item What are the advantages and/or disadvantages of such system from
  system administrator standpoint?
\end{itemize}

The proposed architecture solution presented in chapter
\ref{architecture} will go through a red and blue team inspection,
complying with the QuERIES model in chapter \ref{analysis}.

\subsection{QuERIES} \label{queries}
QuERIES model consists of number of steps that

\begin{enumerate}
  \item model the problem - by conducting a risk assessment of the
    attack surface and the value of the possible intrusion
  \item model the possible attacks - build an attack graph of
    intruding though vulnerabilities or other means
  \item quantify the models - by conducting a controlled red team
    attack and provide quantified results for the said attack
  \item use the results - use blue team methodologies to provide
    increased protection against the exposed problems
  
\end{enumerate}

First, risks are assessed of the attack surface due as a blue team
task. It's very important for blue team to know what are the most
critical points of the attack surface, and it's also used as the base
for quantitative analysis. Value of the intrusion can also be used for
the reward model for analysis. As seen in figure \ref{queries}, the
methodology is applied \textit{iteratively}, i.e the steps are
repeated as many times as needed for the system to be secure.

Modeling the possible attacks is a task for the red team – by
constructing an attack graph, the opposing forces have a plan, which
can be used as a template for analysis. In this thesis, the models are quantified with the use of time
framing. Both teams have limited amount of time to conduct their
tasks, and probability for succeeding a certain task is calculated
with formula

\[ \frac{t_e}{t_t} \]

where \(t_e\) stands for elapsed time and \(t_t\) for maximum time
that can be used which is the same for all tasks.

\begin{figure}[t!]
\centerline{\includesvg[width=0.50\columnwidth]{latex/kuvat/queries.drawio.svg}}
\caption{The QuERIES methodology is used as a reference flowchart for
  evaluation of security. \cite{hughes2013quantitative}}
\label{queries}
\end{figure}
\subsection{QuERIES as a central methodology} \label{queriesasmethodology}

QuERIES draws inspiration from computer science, game theory, control
theory, and economics, thus is a complex answer to a complex
question. It is stated that it can be used as an alternative to
popular methodologies such as red teaming or black-hat analysis used
commonly in risk-assessment. \cite{carin2008cybersecurity}

QuERIES is proposed to have potentially significant usage in DoD
(Department of Defense) and in private sector
\cite{carin2008cybersecurity}. Initial testing of QuERIES in
small-scale, realistic scenarios presented by Carin et al. suggest
that the methodology can in fact be used as to improve risk-assessment
more complex settings \cite{carin2008cybersecurity}. This thesis
follows similar steps: first the QuERIES methodology is used to assess
risks and then they are generalized with strict constraints in mind.

As stated by Hughes and Cybenko the result of QuERIES isn't binary:
the attacker must think about the most optimal timeframe to stop the
operation \cite{hughes2013quantitative}. The strategy uses
\textbf{open} and \textbf{closed} loop decision algorithm for deciding
when to stop trying. Closed loop decision algorithm constantly
evaluates when is the optimal time to stop trying and open loop means
that the system has pre-defined goals for evaluation
\cite{carin2008cybersecurity}. As as positive side-effect by gaining
the probabilities through red-team evaluation the system is
thoroughly tested and improved. This is a valuable metric, providing
insight on how does the value of the system transition in the worth of
breaching it in certain time, thus reflecting the true cost of the
attack. As stated by Hughes and Cybenko, if the value is high enough
related to the value of the gained value, they will less probably
perform the attacks \cite{hughes2013quantitative}.

In this thesis, the value of the whole system architecture is
defined as 1, constituting a value of holded intellectual
property (IP). Costs are defined as fractions of the value, deflating 
0.1 every hour. This is contrary to the original paper, where the
value of the intellectual property has a certain value of 30 000\$ and
the value for the possible intruder is defined as 60\$ per hour
\cite{carin2008cybersecurity}. This divergency is due to the fact
that it's impossible to define a certain value for our system,
and it also provides clarity, as it's ergonomic to see how attack
estimates relate to the value of the intellectual property. 

\subsection{Partially observable Markov decision process}

Lastly, using the results of the POMDP can be used to increase the
protection against the discovered problems
\cite{carin2008cybersecurity}. The original QuERIES methodology used
economic models for estimating POMDP parameters, instead of
calculating them manually \cite{carin2008cybersecurity}. We gain the
parameters from POMDP by calculating state transitions with given
observations. Attacks and defenses are quantified through a partially
observable Markov decision process (POMDP) which contains these six
steps:

\begin{enumerate}
    \item Define possible states the system can be in
    \item Define the actions the system can take
    \item Define the possible observations the system can take
    \item Define the transition probabilities of the system
    \item Define the observation probabilities of the system
    \item Rewards: guide the system towards the desirable actions and
      states.
\end{enumerate} \cite{hughes2013quantitative}

A POMDP is used widely in this kind of applications, as both blue and
red team have only partial observations in relation to the system
\cite{mcabeeMarkov}. The blue team can't be completely certain that
the system is secure and the red team cannot perform fully reliably as
systems and environments differ from each other. As mentioned in the
previous section, the focus of the QuERIES analysis is the time to
reverse-engineer the system, thus emphasizing the importance of only
partial observations.

In the next section the architecture presented in chapter
\ref{architecture} is improved using the applied QuERIES methodology. An
attack graph is constructed and POMDPs are utilized. The result will
provide information about the architectures security, and it will
undergo several iterations of the QuERIES pipeline with each improving
the overall security of the system.
