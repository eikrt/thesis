\chapter{Declarative vs. Imperative systems} \label{imperative}

There have been different approaches to declarative modeling of
systems design. Endres et al. compares declarative and imperative
systems from a cloud computing standpoint, and collects systematic
information on what are the strengths and weaknesses of TOSCA, IBM
Bluemix, Chef, Juju, and OpenTOSCA \cite{endres2017declarative}. Van
der Burg and Eelco Dostra use NixOS as a solution for declaratively
distributing into cloud, executing integration and system tests
\cite{van2010declarative}. Most approaches researched through
literature review focus on distributing to cloud. Distributing to
embedded clearly remains as a niche.

Breitenbücher et al. focus on deploying into embedded and discusses
the challenges an IoT user face when deploying a system \cite{breitenbucher2017declarative}. It's proven
that setting up devices with mandatory scripts and other actions is a
challenging task, when a number of devices should be set up \cite{breitenbucher2017declarative}. Cloud is
something that is useful to be used in tandem with IoT but this thesis
focuses on an \textit{in-premises} reference
solution. 

In this chapter, we focus on comparing different declarative
approaches to the more traditional imperative models, highlighting the
strengths and weaknesses of both. Specifically, examples are provided
to illustrate the limitations often observed in imperative systems,
particularly in terms of reproducibility, scalability and
administration standpoints. Cloud-oriented approaches serve as a prime
reference point for how declarative systems can be effectively
distributed and automated. I can be argued that similiar approaches as those
taken in cloud should be taken in embedded and IoT to increase security through updatability and upgradability.

\section{Imperative systems}

Imperative deployment models base their functionalities through a
process in which the order of events have a critical significance to
the output \cite{breitenbucher2017declarative}. In context of
virtualization, imperative tooling can be used to form a all
activities to be executed, the control flow, their execution order,
and the data flow between them \cite{endres2017declarative}. This kind
of process is best to be used in conjunction with a formalized
workflow or standard such as BPEL \cite{endres2017declarative}. In
contrast, declarative models don't have such specific requirements, as
these models formalize the processes in the configuration files
\cite{endres2017declarative}.

An imperative system provides updatability and modification through a
destructive instrument. Popular imperative package managers, e.g can
remove and overwrite existing files, which leaves the system in an
inconsistent state \cite{dolstra2008nixos}. Different installs have by
nature different states, which causes many problems discussed in this
thesis.

Imperative systems, while popular, have inherent problems regarding
administrative traits contributing to a framework where the underlying
system has \textbf{no traceability}: the implication that
reproducibility is impossible, as changes to a system are not
traced. Nix provides a solution for this problem with its Nix
generation system. With
imperative systems, upgrading is more error-prone than installing from
scratch. This is due to the fact that imperative systems have
\textbf{unpredictable} state, from where the system should migrate to
a predictable state. This causes major issues regarding
upgradability. \cite{dolstra2007purely}

\textbf{The inability to run multiple configurations side-by-side} is
an inherent side effect of a \textit{stateful} system. Declarative
systems don't have this problem: an arbitrary number of configurations
can exist side by side, as the system is defined only by the
configuration, not with the state as a
component. \cite{dolstra2007purely}

\subsection{Debian/Apt}

An example of imperative systems' problematic nature is provided with
the following demonstration. Executing shell command
\begin{lstlisting}
    apt install emacs
\end{lstlisting}
installs a text editor wrapped as a .deb package.  The package emacs
has a dependency, emacs-gtk, which can be removed with command
\begin{lstlisting}
    apt remove emacs-gtk
\end{lstlisting}
Another dependency, emacs-lucid can be removed with command
\begin{lstlisting}
    apt remove emacs-lucid
\end{lstlisting}

we can see that after removing, apt automatically installs emacs-gtk
to avoid breaking the application. The package manager warns:
"emacs-lucid has dependency problems, but removing anyway as you
requested" as shown in figure \ref{deb_remove}. It's also noteworthy,
that the manual page for apt, doesn't say anything about a possible
installation side-effect of a package removal command
\cite{ubuntuUbuntuManpage}.  We could forcibly remove the package by
invoking
\begin{figure}[H]\label{dpkgsnippet}
\begin{lstlisting} 
    dpkg --remove --force-depends emacs-lucid
\end{lstlisting}
\end{figure}
% kuva distribuutioiden distribuutioista
\begin{figure}\label{deb_remove}
\includegraphics[scale=2.0]{latex/kuvat/cropped_apt_output.jpg}
\caption{Terminal output from a Debian system when installing an Emacs package.}
\end{figure}
thus leaving the system in an unreliable state. Dpkg is a low-level
tool associated with apt, and doesn't automatically handle dependency
resolutions or further package relations
\cite{thiruvathukal2004gentoo}. What happens if we had a large number
of devices, in-premises or cloud where all system commands are done
imperatively? We would have a large number of devices that differ from
each other, because as shown, the order of commands affect the state
of the system. Time is also a factor that
causes systems to diverge, as packages are not up-to-date by
default. Invoking 
\begin{figure}[H]\label{aptupdate}
\begin{lstlisting} 
  apt update
\end{lstlisting}
\end{figure}
updates the local repositories to match the download mirrors. If by
technical reasons or possible user error this command is conducted in
the wrong order, there will be divergent systems.

Implementing a deployment model with only Debian would be a gruesome
task, as the order of events which occur during the setup phase is
critical. As presented by Endres et al. a formalized workflow graph
would be needed to set up a reliable system. However, a Debian could
be used as a host to user-space application deployment, such as
Bluemix or Chef, where common DevOps practices can be used
\cite{endres2017declarative}.


%% \subsection{Gentoo/Portage} \label{gentoo}

%% Gentoo Linux is bundled with the package manager Portage, which
%% consists of two components: ebuild and emerge, which have similar
%% relation as dpkg and apt. Portage is primarily source-based package
%% manager; ebuild builds and installs packages from source and emerge
%% resolves dependencies and handles other related issues. Portage's
%% flagship feature is its ''USE-flags'', a mechanism that enables
%% compiling source files with or without specific features. For example,
%% a make.conf file in /etc/portage could have USE-flags specified as:
%% \begin{figure}[H] \label {gentoosnippet1}
%% \begin{lstlisting}
%% USE=''-X wayland''
%% \end{lstlisting}
%% \end{figure}
%% and the sources wouldn't, if possible, compile with X11 support, but
%% would with Wayland. This results in smaller binary sizes, increased
%% performance and enhanced security through package minimalism. When
%% there's less dependencies and installed programs, there's also less
%% attack surface \cite{wang2017network}.

%% Gentoo's portage system allows sharing binaries and sources with
%% rsync, which would be a useful feature in a server-client model,
%% similar to architecture proposed in chapter \ref{architecture}, where updates are centralised
%% \cite{thiruvathukal2004gentoo}. An architecture with imperative
%% components isn't proposed in this thesis, but if it would be, a
%% suitable candidate would be something that uses Gentoo Linux as it's
%% foundation.


\section{Declarative systems} \label{declarativesystems}

Presented problems in chapter \ref{imperative} can be solved using
package manager that is reproducible, reliable and atomic. Package
installs in Nix are in isolation from each other so that they don't
have conflicting effects which results packages being predictable and
assures they work coherently even if underlying install is
different. As the packages are declared in a single set of
configuration files, it's trivial to reproduce the system in a
different environment. The demonstrated effect in snippet
\ref{dpkgsnippet} was a problem due to lack of isolation. When
dependencies are scattered in the system instead of declared
explicitly in a installed package, a faulty state could be
achieved. Nix assures, that these kind of problems are out of the
question. A result of this is that in a Nix system installs of same
program can reside side-by-side with varying versions
\cite{dolstra2008nixos}.

As presented by Endres et al, systems can be declared, even if the
underlying infrastructure is imperative by nature
\cite{endres2017declarative}. This thesis focuses on purely functional
methodologies which fix the most prevalent issues compared with
imperative models. Tools such as Chef focus on deploying on a
imperative system, which causes an inherent problem with cohesion in a
system that should work regardless of the underlying machine or
network. Alternative deployment tools are discussed in section
\ref{nondeclarative}.

One benefit from Nix is its lightweight tendency to enable system
tests. Integrating system tests with a Debian system would require a
considerable amount of work, as setting up such system needs a lot of
configuration and executing commands in a correct order
\cite{van2010automating}. The mentioned distribution in the
previous subsection, Debian, definitely fits in an imperative
deployment strategy but the requirement for explicit detail of every
step would be error prone even for a seasoned administrator
\cite{breitenbucher2017declarative}.

It's also noteworthy that many imperative package managers don't
support rollback mechanisms. If the Nix configuration file is changed
and the system is rebuilt with command
\begin{lstlisting}
nixos-rebuild switch
\end{lstlisting}
the previous state could be recovered by
\begin{lstlisting}
nix profile rollback
\end{lstlisting}
This is an important feature as the Nix configuration files control
the whole system, they can also leave the system in an undesired
state. Nix switches between \textit{profiles}, which is a way to
provide different configurations for different user environments as
shown in figure \ref{userenvs} and provide atomic upgrades and
rollbacks. \cite{nixosNixOSManual}

A fundamental component of the ecosystem is Nix, a domain-specific
language designed for configurations distinguished by its functional
nature and lazy evaluation. The concept of purity is central to Nix,
where values remain unchanged throughout computation, and every
function consistently yields the same output regardless of input
\cite{dolstra2013charon}. The security implications of using Nix
vs. an imperative system is discussed in the next chapter.

\subsection{Non-declarative components} \label{nondeclarative}

Declarative distributions such as Nix can't do everything in the
system in stateless manner. Some components of the system, such as
databases must have a distinct state, which can't be practically
declared with package manager apart from initial configurations
\cite{van2013reference}. Home directories can vary as much as the
system administrator desires. For example, a configuration file for
text editor vim is usually declared in the file
/home/<user>/.vimrc. Nix provides multiple ways to perform the whole
configuration process from the Nix configuration files. One way is
declaring the desired .vimrc in the Nix configuration, as in the
following snippet:

\begin{lstlisting}
{
  environment.systemPackages = [
    (pkgs.vimConfigurable.customize {
      vimrcConfig.customRC = ''
        " arbitrary vim config
      '';
    })
  ];
}
\end{lstlisting}
Nix provides also provides ways to fetch content to the system from
remote URLs, and if the administrator doesn't want the system to
remain ''pure'', they can build the system by
\begin{lstlisting}
  nixos-rebuild switch --impure
\end{lstlisting}
This results the system having mutable components, which can be
desirable from an accessibility point of view, but can result in
unpredictable behaviour if the impure components are modified. Purity means that the components are read-only and
immutable \cite{dolstra2010nixos}.

User environments (Nix profiles) can be used so that for different
needs, or for different users there are multiple environments in which
the user can operate as shown in figure \ref{userenvs}. User
environments are a successor to the concept, where installed programs
either reside in
\begin{lstlisting}
/usr/bin
\end{lstlisting}
or
\begin{lstlisting}
/usr/sbin
\end{lstlisting}
etc. or have a symbolic link to
the said directories. They can be figured as trees of symbolic links
that reside also in the Nix store hence referred packages are called
''activated packages''. The installed programs reside usually in
\begin{lstlisting}
  /nix/store
\end{lstlisting}. \cite{dolstra2008nixos}

\begin{figure}[t!]
\centerline{\includesvg[width=1.0\columnwidth]{latex/kuvat/symlinks.drawio.svg}}
\caption{Relations between different user environments and installed
  programs \cite{nixosUserEnvironment}.}
\label{userenvs}
\end{figure}

There exists continuous build and integration services, such as Hydra,
which include Nix-compatible support for handling runtime
configuration and tools, such as Disnix and Charon \footnote{Charon is
now called NixOps \cite{githubNixNixpkgsNixOS}}, which focus on
setting up complementary infrastructure. Van Der Burg presents these
new tools to replace Cfengine, Puppet and Chef, which execute
operations in convergent manner, meaning that they capture what
changes should be done to the machines in a specified
network. \cite{van2013reference}

These approaches have two central problems: imperative nature of
handling environment difference, and inability to guarantee
configuration compatibility with a machine. Disnix is a Nix derivative
that can overcome these challenges by separating logical properties
from physical, and by capturing the essential aspects which form a
system. \cite{van2013reference}

\subsection{Home manager and flakes}

Nix environments can be build from a single configuration.nix file,
but there are two significant configuration tools for managing Nix
systems: home manager and flakes. Home manager is an extension for
managing user profiles with a declarative Nix syntax
\cite{nixcommunityHomeManager}. Home manager has problems with atomic
rollbacks and for this reason they are not used in this thesis'
examples \cite{nixcommunityHomeManager}.

Flakes are experimental feature of Nix, providing environments, where
dependencies are pinned in a lock file, further improving
reproducibility of Nix systems. A flake is a
file system tree, which contains a root directory with the Nix file
specification called flake.nix. The usage of flakes is a good method
for organising different environments within a Nix system, where it
can consist of multiple flakes. Flakes are an experimental
feature, thus out of this thesis' scope. \cite{nixosFlakesNixOS}
% referoi manuaalia
\subsection{Ease of updates}

Updating is easy and riskless with NixOS due to atomic
rollbacks. Nix handles software providing through something called
channels. A channel is a set of latest Git commits in a Nixpkgs
repository, where they are divided to stable/unstable and
large/small. channels. Unstable channels (large and small) have the
latest commits on a rolling basis, but include less conservatively
checked functionalities. Stable channels are submitted through a
version number (e.g. 23.11), where a new release is published every
six months. Large channels contain a full set of
Nixpkgs binaries, when small include a subset. If a system
administrator decides to submit to small channel, they have more
recent updates at their disposal, but have to resort to compiling some
needed packages from source. \cite{nixosChannelsNixOS}

Updating a Nix system is just a manner of invoking command

\begin{lstlisting}
    sudo nix-channel --update
\end{lstlisting}

and, if stable release is chosen, updating the system.stateVersion
from the configuration.nix file \cite{nixosNixOSManual}. Nixpkgs is a
repository of working Nix packages using a continuous integration
service called Hydra. Hydra evaluates the needed Nix expression of a
package, and ensures its functionality. \cite{nixosNixOSManual}

It's apparent that purely functional, declarative approaches aren't as popular
as imperative systems due to the need of steep learning curves and
obscure syntax. This can be seen as historical payload: imperative
models have been in use a much longer time than purely functional
approaches, such as Nix.

In this chapter we revealed, that declarative systems have inherent
strengths in both deployment strategies and system upgrading. The next
section brings up the security viewpoint of declarative systems
specifically in an embedded Linux setting.
