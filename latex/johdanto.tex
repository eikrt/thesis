\chapter{Introduction} \label{johdanto}

A Linux distribution is a bundle of the Linux kernel and a set of
software products called packages \cite{gnuPackagesx2014}.  A package
manager is an instrument that handles building packages either
from source or pre-built binaries, resolving build-time and run-time
dependencies of packages and installing, removing and upgrading
packages in user environments \cite{gnuPackagesx2014}. Every Linux
device must handle its installed programs with their dependencies and
configurations either imperatively or declaratively. Overwhelmingly
large portion of Linux distributions fall in the first category
\cite{dolstra2008nixos}. Both imperative and declarative distributions
have their strengths and weaknesses from administrative and security
standpoints.

An imperative distribution updates and modifies packages destructively. Apt,
apk, dnf and zypper are some of the popular package managers \cite{dolstra2008nixos}. Imperative package
managers can remove and overwrite existing files which leaves the
system in an inconsistent state. Different installs have
different states by nature which causes many problems discussed in this
thesis. As files are cross-modified through packages with package
managers such as apt, upgrading can be disastrous as such distributions
don't support atomic rollback capabilities \footnote{Some Linux
distributions using btrfs filesystem can perform a snapshot and
rollback, meaning that a previous state of a system can be recovered without issues \cite{opensuseSystemRecovery}}. Due to the unpredictability,
the result can be a partially or completely broken system
\cite{dolstra2008nixos}.

The reference imperative Linux distribution in this thesis is Debian
with its default package manager apt, due to its popularity and
relative simplicity. The reference declarative distribution is NixOS
with its partial namesake package manager Nix which provides
declarative configuration of the whole system including the Linux
kernel\footnote{There exists an experimental project that has
succeeded with BSD interoperability
\cite{githubGitHubNixosbsdnixbsd}}. Nix is configured by the Nix
programming language which is inspired by purely functional languages
such as Haskell. Declarative systems are those whose operation is based on a set of configuration files that define the system. \cite{van2013reference}

The reference architecture depicted in chapter \ref{architecture} is
based on Nix as it is the most popular purely functional and declarative
Linux distribution with over 100 000 packages
\cite{nixosNixOSSearch}. Architecture in this context refers to a 
set containing clients, server and the used hardware. The term
system means the implementation of the architecture, especially
referred in chapter \ref{analysis}. Depending on the context, system
can also mean a set of procedures, e.g. a Linux system. 

Another good distribution for the architecture would have been Guix, with over
28 000 packages \cite{gnuPackagesx2014}. Guix has some improvements
over Nix, including richer and more extensible programming environment
with a Lisp-dialect configuration language, Scheme
\cite{courtes2021deploiements}. NixOS remains as the distribution of
choice, as the number of packages is greater and general support is
found to be better.

This thesis discusses how declarative distributions can be used as an
improvement over imperative distributions. In chapter \ref{imperative}
both approaches to package management are compared followed by an
introduction to embedded security in chapter \ref{embedded}. Chapter
\ref{architecture} presents an example architecture created with
NixOS. In chapter \ref{analysis}, a quantitative research is carried
out revealing strengths and weaknesses of the reference Nix
environment setup. The selected methodology, QuERIES provides
quantitative results that can be used to improve the security of
declarative architectures. Propositions for further research are
covered in chapter \ref{further} and finally, chapter \ref{conclusion}
concludes the thesis.

\section{Research methodologies and questions}

Research methodologies used in this thesis are:
\begin{enumerate}
\item a literature review of central papers on subject themes found
  with prepared search statements
\item an action research using a laboratory setup
\item a quantitative research process using QuERIES methodology
\end{enumerate}

Literature review will be addressed mostly in the next section and in
chapter \ref{embedded}. As this thesis' main research methodology is
quantitative, the gathered data points will be addressed as variables
that are compared with mathematical methods. Quantitative methods are
broken down in chapter \ref{securitystandpoints} and
\ref{analysis}. The central methodology is derived from QuERIES, and
the information security aspect is investigated with CIA-triad. In
chapter \ref{analysis} section \ref{whyqueries} QuERIES is compared
with other metrics which are explained in subsection \ref{resquest}.

Quantitative methodologies are oftentimes used in conjunction with
qualitative methodologies, with both approaches having their strengths and
weaknesses. One drawback of using qualitative methods in security
framework is their inherent subjectivity. For example the Delphi
technique, where a set of opinions is gathered and compared from a
working group, provides subjective substance for a study instead of
objective perspectives \cite{wang2005information}.

The study design in this thesis is \textit{state based}, which refers
to the fact that the research methods focus on different state
transitions, e.g. how probable it is  for an intruder to gain from
partial leverage to a full control of the system. Qualitative research
wouldn't alone satisfy the requirements, as investigating different
state transitions without quantitative methodologies would be absurd
\cite{ramos2017model}.

\subsection{Literature review}\label{litrev}

This thesis has bibliography from 49 sources, most of it gathered with
a carefully prepared search statement. Other sources include manuals,
material for research methods and other relevant material. The search
statements results presented in next subsection, provide a quality base
for action research and analysis.

The literature focuses on four main concepts: embedded systems
security with and without declarative components, imperative systems,
measuring security and Nix. The main goal is to find literature that
combines these concepts to gain platform for comparing different
approaches to support the action research.

Central literature revolves around Nix and multiple texts by Eelco
Dolstra are cited for illustrating the nuances of a Nix
ecosystem. Other declarative approaches are discussed by Endres
et al. and Van der Burg \cite{van2010declarative, endres2017declarative}. These approaches also contain comparison to
imperative systems, which is the central viewpoint in chapter
\ref{imperative}. Combining cyber security with declarative approaches
were discussed by Specht et al. and Kandoi and Artke
\cite{specht2007analysis, kandoi2021operating}.

Discussion from Ravi et al. and Fysarakis et al. handle embedded security
extensively \cite{ravi2004security, fysarakis2014embedded}. The
approaches itself  are too broad for this thesis' scope, so
only the most fitting were selected for use.

The most important literature to measure the security of a system is
by Carin et al. and Hughes and Cybenko \cite{carin2008cybersecurity,
  hughes2013quantitative}. The topics from these papers revolve around
QuERIES, an original approach for measurably improve security. 

\subsubsection{Search statement} \label{searchstatement}

The main search statement for this thesis is: "embedded linux" OR
''declarative'' AND (linux OR *nix) OR deployment OR ''system update''
OR (compare* AND declarative AND imperative AND system*) OR security.

The search statement was prepared to provide as relevant results as
possible for this thesis. The main goal was to include the hypernyms
''embedded linux'', ''linux'' with other terms separated using the
"OR" operator. The subterm (compare* AND declarative AND imperative
AND system*) was chosen to broaden the search to include articles
which compare declarative and interactive systems.

As security is a central theme in this thesis, the term "security" was
included. Systems security
and cyber security materials are also included in the bibliography using
search statement ''systems security OR cyber security''. Separate search
''cia-triad'' and ''partial observable Markov chain'' AND
''cybersecurity'' were used to provide tangible meters for measuring
cyber security. To further back up the research for comparing
different metrics, term ''cyber security metric methodology'' was
searched.

For searching specific material about embedded systems, the search
statement ''embedded AND security'' was used. All searches were done on Google Scholar platform.

\subsection{Research questions} \label{resquest}

The research questions for this thesis are:

\begin{enumerate}
\item How can a declarative system be used to measurably improve the
  security of an embedded system used for displaying public media?
\item What are the advantages and/or disadvantages of such system from
  a system administrator standpoint?
\item How can a system using different Linux distribution switched to use NixOS securely and seamlessly?
\end{enumerate}

Research question 1 is the most important and it traverses through
themes of the whole thesis. The hypothesis is that traditional
imperative embedded device fleets have problems that can be solved
with the use of modern declarative systems. First we aim to gain
information from a specific scenario presented in chapter
\ref{architecture}. Then in chapter \ref{analysis} the gained
information is analyzed and generalized as suitably as possible.

Research question 2 brings up the human element; how can a system
administrator use a new palette of features adequately to provide a more
secure system. Research question 3 handles a situation where
an existing system should be replaced with a NixOS system. How this could
be done securely without risks and preferably easily with existing
tooling, is answered in chapter \ref{architecture} section
\ref{instnewdevices}. The next chapter compares imperative approaches
to declarative approaches and provides insight for understanding the
central differences, the emphasis being on how declarative systems can be
used to solve problems better than with imperative systems.

\subsection{Data collection and analysis}

Data collection is done with simulated red–blue team setup, where both
teams are provided by a time frame where they must conduct a series of
tasks. These tasks are formalized as partially observable Markov chain
parameters, and analyzed with QuERIES methodology. This methodology is
used to gain knowledge and make the system more reliant and better
with multiple iterations. Chapter \ref{analysis} answers research
questions 1 and 2 and provides analysis for the reference
system. Research question 3 is answered in chapter \ref{architecture}
section \ref{instnewdevices}.
