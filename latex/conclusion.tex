\chapter{Conclusion} \label{conclusion}

Nix can be a useful base for issuing a purely functional, declarable
systems while providing an advanced rollback mechanism. Configuring it
depending on the system administrator may be gruesome, but it can
dodge the usual pitfalls of more popular systems. While the usage of
Nix language demands proficiency, it is a solution to distribute,
update and administrate more secure systems. NixOS handles
exceptionally easily features such as Wayland and audio driver
setup. In the end, NixOS is very predictable and the statement "if it
works on one machine, it works on another" proves to be true.

This thesis' problems could be solved with many different solutions,
but the Nix approach is rather unique as it makes reproducible systems
very straight forward. As the system is defined in the
configuration.nix file, there's little that could go wrong even when
underlying hardware varies. The same couldn't be said of mentioned
Debian distribution.

Measuring and improving the architectures solution proved to be
challenging but fruitful. With multiple iterations used with QuERIES,
the architecture developed from insecure to more secure. This is a
testament for applying successfully a certain methodology for a
specific task.

% security implications closing
As far as security implications are concerned, Nix is a good tool for
asserting compliance. In purely functional environment, anything that
needs compliance can be checked from the configuration files. As disk
encryption is trivial and read-only Nix store provides increased
integrity, a more secure system can be achieved also through the
actions of a system administrator. A system administrator must adhere
at to Nix philosophy with the Nix language and as all the configured
parts can be viewed quickly. For example, a question ''what ports are
in use in each client device'' can be answered very quickly and in
precise manner by just viewing the configuration files.

As mentioned, popular embedded Linux distribution creating tools,
Buildroot and especially Yocto have also steep learning curves. They
also lack the ease of updatability provided by Nix, which is a
significant problem regarding security.

The research proved that the purely functional Nix system is easily
updatable and upgradable solution for maintaining a secure system.
While NixOS is a niche Linux distribution, and probably remains as
such, the concepts, however I assume will live on in future. A purely
functional declarative system with atomic rollbacks is something I'll
await to be iterated in the future with further user availability in
mind.
