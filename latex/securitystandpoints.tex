\chapter{Security standpoints} \label{securitystandpoints}

Security is inherently challenging to measure adequately due to its
complex and chaotic nature. Qualitative analysis may also result in
subjective output. As such, no unambiguous standard of measuring
security can be provided \cite{wang2005information}. To overcome these
challenges, several metric systems are compared and the one that
provides the most precise answers specifically for this thesis' cause
is chosen.

In this chapter, a selection process for meters which in part are used
for gaining quantified data, is presented.  When finally quantitative
metrics are gained with the help of two paradigms, GQM (goal, question, metric) and
specifically its superset SMART, a red–blue team layout is erected
using QuERIES methodology. SMART is used in conjunction with the
literature review to reveal the most suitable methodology for this
thesis. SMART is opened more in the section \ref{choosingsecmet}.

Using an exact metric system is important, as this this thesis'
emphasis is on measuring the security of a declarative approach. The
presented architecture in the previous chapter works as an example on
how a declarative system can be improved by reconfiguration, utilising
a red-blue team setup.

\section{A brief history of security metrics}

The history of security metrics begin from Trusted Computer System
Evaluation Criteria (TCSEC) also known as the Orange Book from 1983,
which popularised many terms still in use today, such as
identification, authentication and
authorisation \cite{bayuk2013measuring}.

In the 1990s, when the US National Bureau of Standards (NBS), later
known as the National Institute of Standards and Technology (NIST),
tried to standardise security, it became clear that systems needed to
adhere closely to the definitions outlined in the Orange Book and the
subsequent Common Criteria project \cite{bayuk2013measuring}. Over time, various other standards
gained popularity, such as the System Security Engineering Capability
Maturity Model (SSE-CMM), which serves as a checklist for system
design from the ground up. 

Later it was observed, that systems design is only a part of a
successful security strategy and operational practices played a
bigger role than expected. This was something to be addressed in 1995 NIST
Computer Security Handbook which evolved to provide ground to
combat modern issues \cite{bayuk2013measuring}.

As metrics can be used as a tool for decision making, the strategical
approach of the mentioned publishes is important. It is noteworthy, that
the strategies (the Orange Book, SSE-CMM, etc.) begin to measure
security by compliance to defined ratings \cite{bayuk2013measuring}. Later in 2000s more
mathematical approaches were taken, one which is delved deeper in
section \ref{whyqueries}. 

\section{Choosing security metrics} \label{choosingsecmet}

Security is something that is challenging to measure due to its
complex nature. A GQM (Goal, question, metric) paradigm helps to
choose appropriate metrics: first there must be a set goal to an
organisation, then a formulated question for each goal \cite{papazov2019cybersecurity}. These answers
are then reflected to gain the desired metric. This strategical
approach is perhaps too broad for this thesis' scope, but it aligns well
with a usual organisational strategy. 

A more appropriate tool for this task would be SMART – a set of inputs
to evaluate meter systems' suitability. These inputs describe how
specific, measurable, attainable, relevant and timely the methodology
is \cite{payne2006guide}.

In cyber security, being specific is very important. A common issue
with security meters is, that they either cover too many topics and
are without precise definitions or they are too specific to be
generalised to a broader scope of situations
\cite{wang2005information}. The results of this thesis are crucial to
measure, as the research focuses on system states and aims to assess
the outputs produced by state transitions.

The concept of attainability is important in this context because this
thesis has a narrower scope compared to a large
organisation. Therefore, the proposed setup must be evaluated to
ensure that the metrics' goals are achievable. Relevance has to do
with risk assessment and how important it is to measure something related
to its value. Risk assessment is explored in detail in chapter
\ref{analysis} section \ref{modprob}. Time-bounding signifies the
importance of time as a meter; a system that can be penetrated in a
minute can certainly be seen as weaker than a system that takes years
to be compromised.

\section{Measuring security} \label{measuringsecurity}

In this section, different methodologies and perspectives gained
through literature review for cybersecurity are discussed, and
potential methodologies are compared to gain the most adequate metric
system for usage through SMART process \cite{payne2006guide}.

Security metrics can be categorised into four themes:
\textbf{system vulnerabilities}; measuring vulnerabilities can be
applied to user-related, interface-induced, password, and software
vulnerabilities \cite{pendleton2016survey}. Users are constantly at risk of threats such as
phishing attacks or malware infections, where the user of any system
becomes the primary attack vector. Interface-induced vulnerabilities refer to attack vectors associated with open ports
and endpoints.

Password vulnerabilities refer to instances where a password can be
cracked through computational methods \cite{pendleton2016survey}. This is fairly easy to measure,
as it is possible to estimate the time required to crack a password or
assess its vulnerability using statistical password
guessability. Software vulnerabilities are a common cause of security
breaches. These vulnerabilities can be measured and estimated based on
past exploitations. A key metric in this context is the time taken to
patch a software vulnerability. 

\textbf{Defense} measures can be applied to strengthen reactive,
preventive, proactive and overall defenses. Reactive measures include
blacklisting which is a lightweight mechanism to prevent e.g. for preventing botnet to harm
the protected system by blacklisting IP-addresses related to the
botnet. For measuring defence, the reaction time is essential and
most importantly, it is a gained meter to measure preventive
defense. Blacklisting can also be used as a preventive and a proactive
measure, as a pre-filled blacklist can be used with desired
parameters \cite{pendleton2016survey, ramos2017model}.

Overall defenses can be measured with the combination of all defensive
measures and with the use of penetration testing in a red–blue team
setup. Penetration testing aims to gain a result, also known as
penetration resistance, which is a meter indicating cost or time that
the red team must spend in case of a successful system
compromisal \cite{pendleton2016survey, ramos2017model}.

\textbf{Threats}: zero-day vulnerabilities can be measured from two
perspectives: lifetime of zero-day vulnerability and the number of
nodes that are compromised as a result. Malware spreading can be
traced with the parameter infection rate, which is defined as infected
node per a time unit. Attack evasion is assessed using either the
obfuscation prevalence metric or the structural complexity
metric. These metrics offer insights into the obfuscation of acquired
samples, such as through encryption, or the complexity of the target
system, which is measured by its runtime \cite{pendleton2016survey,  ramos2017model}.

\textbf{Situations} can refer to the security state, security
incidents, and security investments. The security state encompasses
various parameters, such as the incident rate and the blocking
rate. Security investments measure the percentage
of the budget allocated to security and the return on those
investments \cite{pendleton2016survey}. 

\section{Methodologies in comparison} \label{whyqueries}

Cybersecurity metrics based on quantified
mathematical models, which are prevalent for this thesis exist today. Three
different methodologies are discussed, and one is picked for measuring
the security of this thesis' architecture implementation. All the following metric
systems are \textbf{measurable}. However, some fit better in relation with \textbf{time-related} and \textbf{relevance} axes.

The literature review provided three central methodologies from different perspectives. Complex mathematical models
presented by Alshammari et al. are too broad
\cite{alshammari2009security}. This thesis' scope is limited and this
methodology would fit better in a wider cyber security setting.

A methodology based on object-oriented thinking and UML diagrams is
suitable in many contexts. However, as mentioned in the paper, this
measurement methodology is used to compare similar alternative designs
\cite{alshammari2009security}. As noted, the focus of this thesis is
not comparative, since all critical comparisons have already been made
in chapter \ref{imperative}.

The Hidden Markov models presented by Wang et al. are closely aligned
with the end goal of this thesis \cite{wang2010framework}. The
time-related aspect is suitable, as the Hidden Markov model
incorporates a time parameter. However, the issue lies in the
generality of the methodology, and a more specific approach would be
better suited for this thesis. Nevertheless, we apply a similar
approach to that of Wang et al. for calculating the POMDP parameters.

The last and the most fitting methodology would be the one presented
in papers by Carin et al. and Hughes et al.
\cite{carin2008cybersecurity, hughes2013quantitative}. The QuERIES
methodology is delved deeper in section \ref{querieschapter} and its
time-related, relevance and specific axes are a near-perfect match for
our goals as the model itself is relatively simple and provides
shifting probabilities from states, which serves this thesis' study
design well.

\section{Quantitative metrics} \label{quantitativemetrics}

Carefully selecting a metrics system includes asserting our goals and
questions. Our goal is to discover this thesis' architecture proposals'
tenacity in a simulated setting. The main goals reside in this thesis'
two research questions:

\begin{itemize}
\item How can a declarative system be used to measurably improve the basic
  security needs of an embedded system used for displaying public
  media?
\item What are the advantages and/or disadvantages of such system from
  system administrator standpoint?
\end{itemize}

The proposed architecture solution presented in chapter
\ref{architecture} will go through a red and blue team inspection,
complying with the QuERIES model in chapter \ref{analysis}.

\subsection{QuERIES} \label{querieschapter}
QuERIES model consists of number of steps that

\begin{enumerate}
  \item model the problem - by conducting a risk assessment of the
    attack surface and the value of the possible intrusion
  \item model the possible attacks - build an attack graph of
    intruding though vulnerabilities or other means
  \item quantify the models - by conducting a controlled red team
    attack and provide quantified results for the said attack
  \item use the results - use blue team methodologies to provide
    increased protection against the exposed problems
  
\end{enumerate}

First, the blue teams' task contains the risk assessment of the attack
surface. It is particularly important for the blue team to acknowledge
the most crucial points of the attack surface, that is also used as
the base for quantitative analysis. As seen in figure
\ref{queries}, the methodology is applied \textit{iteratively}, i.e
the steps are repeated as many times as needed for the system to be
secure.

Modeling the possible attacks is a task for the red team – by
constructing an attack graph, the opposing forces have a plan which
can be used as a template for analysis. In this thesis, the models are quantified with the use of time
framing. Both teams have a limited amount of time to conduct their
tasks and the probability for succeeding a certain task is calculated
with the following formula

\[ \frac{t_e}{t_t} \]

where \(t_e\) stands for elapsed time and \(t_t\) for maximum time
that can be used which is the same for all tasks.

\begin{figure}[t!]
\centerline{\includesvg[width=0.50\columnwidth]{latex/kuvat/queries.drawio.svg}}
\caption[QuERIES as a flowchart]{The QuERIES methodology is used as a reference flowchart for
  evaluation of security. \cite{hughes2013quantitative}}
\label{queries}
\end{figure}
\subsection{QuERIES as a central methodology} \label{queriesasmethodology}

QuERIES draws inspiration from computer science, game theory, control
theory and economics, thus is a complex answer to a complex
question. It has been stated, that it can be used as an alternative to
popular methodologies such as red teaming or black-hat analysis, used
commonly in risk-assessment \cite{carin2008cybersecurity}.

QuERIES is proposed to have potentially significant usage in DoD
(Department of Defense) and in the private sector
\cite{carin2008cybersecurity}. Initial testing of QuERIES in
small-scale and realistic scenarios presented by Carin et al. suggest
that the methodology can in fact be used as to improve risk-assessment of
more complex settings \cite{carin2008cybersecurity}. This thesis
follows similar steps: first the QuERIES methodology is used to assess
risks followed by generalisation with strict constraints in mind.

As stated by Hughes et al. \cite{hughes2013quantitative} the result of QuERIES is not binary: the
attacker must think about the most optimal timeframe to stop the
operation. The strategy uses
\textbf{open} and \textbf{closed} loop decision algorithms for
deciding when to stop trying. Closed loop decision algorithm
constantly evaluates when is the optimal time to stop trying and open
loop means that the system has pre-defined goals for evaluation
\cite{carin2008cybersecurity}. As a positive side-effect, by gaining
the probabilities through red-team evaluation, the system is
thoroughly tested and improved. This is a valuable metric, offering
insight into how the value of the system changes over time in terms of
the cost of breaching it, thereby reflecting the true cost of the
attack. As stated by Hughes and Cybenko \cite{hughes2013quantitative}, if the value is high enough
related to the value of the gained value, they will
perform the attacks less likely.

In this thesis, the value of the whole system architecture is
defined as 1, constituting a value of holded intellectual
property (IP). Costs are defined as fractions of the value, deflating 
0.1 every hour. This is contrary to the original paper where the
value of the intellectual property has a certain value of 30 000\$, and
the value for the possible intruder is defined as 60\$ per hour
\cite{carin2008cybersecurity}. This divergency is due to the fact
that it is impossible to define a certain value for our system.
It also provides clarity as it is ergonomic to see how an attack
estimates the relation to the value of the intellectual property. 

\subsection{Partially observable Markov decision process}

Lastly, using the results of the POMDP can be used to increase the
protection against the discovered problems
\cite{carin2008cybersecurity}. The original QuERIES methodology used
economic models for estimating POMDP parameters instead of
calculating them manually \cite{carin2008cybersecurity}. We gain the
parameters from POMDP by calculating state transitions with given
observations. Attacks and defenses are quantified through a partially
observable Markov decision process (POMDP) which contains the following six
steps:

\begin{enumerate}
    \item Define possible states the system can be in
    \item Define the actions the system can take
    \item Define the possible observations the system can take
    \item Define the transition probabilities of the system
    \item Define the observation probabilities of the system
    \item Rewards: guide the system towards the desirable actions and
      states.
\end{enumerate} \cite{hughes2013quantitative}

A POMDP is used widely in these kinds of applications, as both blue and
red teams have only partial observations in relation to the system
\cite{mcabeeMarkov}. The blue team cannot be completely certain that
the system is secure, and the red team cannot perform fully reliably as
systems and environments differ from each other. As mentioned in the
previous section, the focus of the QuERIES analysis is the time to
reverse-engineer the system thus emphasising the importance of only
partial observations.

In the next section the architecture presented in chapter
\ref{architecture} is improved using the applied QuERIES methodology. An
attack graph is constructed and POMDPs are utilised. The result will
provide information about the architecture's security, and it will
undergo several iterations of the QuERIES pipeline with each improving
the overall security of the system.
