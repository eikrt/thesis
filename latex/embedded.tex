\chapter{Embedded system security} \label{embedded}

According to Serpanos et al. the use of embedded devices can be
divided into four fields: industrial systems, nomadic environments,
private spaces and public infrastructure
\cite{serpanos2013security}. This thesis' focus is public
infrastructure, specifically information screens in a public
environment. Implementing security mechanisms and policies is
essential for information screens to function securely from both
organizational and technical viewpoints. Implementing those policies
and assuring compliance is trivial with declarative approaches with
increased benefits from reliability perspective.

Embedded systems are distinct from other type of systems due to their
varying nature ranging from programmable logic controllers (PLC), to
larger systems, such as servers or
routers. \cite{fysarakis2014embedded}. An usual embedded device
conducts a specific task, and possibly demands networking
capabilities. Working with embedded is typically with a limited set of
resources, and it demands careful design when a multitude of features
are needed, but the underlying systems have limited computing
power. Maintaining and upgrading devices to meet the continuous need
of security updates. Even services such as SSH have had history of
vulnerabilities, which prove that upgradability is a fundamental base
of a secure system \cite{secopsolutionHistorySecOps}. I argue that the
security aspect of embedded devices could be improved significantly
with the use of declarative systems as seen in chapter
\ref{imperative} section \ref{declarativesystems} and in the following
section \ref{nixosassolution}.

Embedded devices demand precision and security, as their function may
be very critical for variety of safety reasons, e.g in automotive
industry or healthcare applications \cite{turab2019secure, fysarakis2014embedded}. Reliability is a defining requirement
for number of embedded applications; a pacemaker that doesn't function
all the time reliably is completely useless. While a declarative
solution itself can't fulfill all security needs, it definitely could
improve the \textit{reliability} of such systems.

As stated by Fysarakis et al. implementing access control is essential
for any system to prevent unauthorized access
\cite{fysarakis2014embedded}. Implementing complex access control is
trivial with Nix, as the configuration files denote completely which
user has accesses to which resources. Access control in a modern day
embedded environment could be hard to implement in traditional
imperative systems, as scaling an access control system which spans
multiple devices and changing environments would require a lot of
manual intervention. This is definitely one of strengths of
declarative approaches: scalability is never an issue when a
centralized configuration defines the systems. A declarative approach
is often taken in the cloud as stated in section chapter
\ref{imperative} section \ref{declarativesystems}, but implementing declarative overlays
definitely needs work in the embedded field.

Implementing policies information security perspectives using a policy
modeling standard, CIA-triad is discussed in section \ref{imperativeanddeclarative}, and NixOS is reflected with
the use of the triads axes. Dolstra states that Nix is policy-free
meaning that it contains a set of mechanisms which allow policies to
be constructed with and not the other way around \cite{dolstra2004nix}.

Embedded being a broad field, in this thesis devices are limited to
those which can run Linux kernel and provide the most basic networking
capabilities. These cover architectures i686, x86\_64, arm64 supported
by NixOS. PLCs and microcontrollers are outside of scope as NixOS
needs a functional Linux kernel and a specific architecture to work.

\section{Common embedded pitfalls}

Common issues regarding embedded devices are their lack of updates,
weak data integrity, and the multitude of features
\cite{kemmerer2003cybersecurity, fysarakis2014embedded}. For
example, a toy teddy bear may have a audio recorder, data transfer
capabilities and ability to geolocate itself. These kind of devices
may lack firmware or software updates, and the data-transfer may be
insecure.

A solution for secure data transfer would be TLS-encrypted messaging
between clients. This could be achieved with MQTT-protocol, but
configuring certificates is extra effort. Multitude of features is a
definite security problem, as the user may not be aware of them at all
times. In an increasing global world, importing embedded devices from
unreliable sources can prove to be a security issue. The household
items may or may not adhere to latest security
compliance. \cite{fysarakis2014embedded}

Attack surface of embedded systems in general range from physical
access to network and geolocation problems. One way of manipulating a
device, apart from directly gaining access to the operating system,
are side channel attacks. Analysing the power or electromagnetic
properties of device input/output can be used to determine critical
aspects of a device, e.g key lengths or algorithms of security
measures \cite{fysarakis2014embedded, serpanos2013security}. Attack surface may used to gain access, or performing denial
of service attacks. Geolocating is both a privacy and security issue,
as location data may be used to trace identities of device users,
which can lead to e.g blackmailing, physical intrusion or other
means \cite{fysarakis2014embedded}. This means that the principals of
this thesis' scope could theoretically be targeted with such malicious intents. 

Embedded systems have problems regarding monitoring and system
administration. It's very different to have home automation system
with less than 20 nodes, than to have public transport embedded fleet
in a big city with 2000 nodes. As the number of devices grow, so does
the challenge of monitoring and administrative tasks. Home automation
has usually one person dedicated to the task: the home owner. The
hypothetical setup with 2000 devices has an exponential growth of
problems. Monitoring should be trivial to automatize (e.g by using
tools like Prometheus), but administrative tasks are harder to
automatize, due to tasks being potentially very challenging even for
dedicated system administrators. This is where Nix comes to play, as
updating thousands of devices becomes trivial.


\section{NixOS as an embedded solution} \label{nixosassolution}

Declarative systems have advantages over imperative systems in
reliability and safety aspects due to two things:

\begin{enumerate}

\item rollout and rollback are equally trivial tasks
\item desired configuration can be tested in a sandbox environment
  
\end{enumerate}

The first item makes it more accessible to manage a rollout strategy,
as the rollout/rollback can be done multiple times or executed
completely in a replicated sandbox environment, as stated in item
2. Simpler and more straight forward practical steps give space for 
easier strategical planning. \cite{kandoi2021operating}.

Kandoi et. al argue that with declarative systems, it should be nearly
impossible to misconfigure in the first place the system and if faulty
state is achived, a simple rollback could undo the changes
\cite{kandoi2021operating}. As stated in chapter \ref{imperative}
section \ref{declarativesystems}, it's definitely possible
to achieve faulty systems with Nix. I argue that these problems can be
mitigated with a well thought rollout/rollback strategy. 

Updatability is possible with many different platforms, but it's a
problem when updating is a sole duty of a consumer, who may or may not
have the adequate knowledge how or why they should update their
systems. Lightweight updatability comes out-of-the-box with Nix, and
that is something that inherently should make it more secure. Consumer
products, however are out of this thesis' scope.

Nix is a double edged sword for system administration tasks. On 
one hand, it has a steep learning curve, but on the other hand it
can make tasks that could be very challenging with traditional
systems, trivial. In a well built Nix ecosystem security actions such
as updating or modifying user or kernel space can be used to enhance
security and in such system, any changes could easily be replicated to
multiple devices, without the need for manual intervention.

Some other clear disadvantages for NixOS in embedded use is the fact
that a purely functional, declarative system inherently must use disk
space more than it's imperative counterparts. In the worst case
scenario, if one derivation of a system takes up 1Gb of space, when
making changes, the resulting system will need 2Gbs of space. The
worst case scenario rarely occurs, but due to Nix's indestructive
nature, this formula of disk space demands has to be considered in an
embedded setting. \cite{dolstra2007purely}

\section{Imperative and declarative systems from CIA-triad approach} \label{imperativeanddeclarative}

CIA-triad can be used as a tool to show conflicts between different
points of information security interests. It consists of three meters:
confidentiality, integrity and availability as seen in the figure
\ref{ciatriad}. Confidentiality can be seen as superset of
privacy. Confidential data is classified with technologies such as
data encryption and user privileges. Integrity means that the data has
not been tampered with, and remains untouched by unauthorised parties
while it's in transit or stored e.g in a server. A way of providing
integrity is checking hashes of downloaded files. Availability is a
user viewpoint to the accessibility of the system. When
confidentiality and integrity are pushed to the extreme, availability
aspect suffers, e.g when a service enforces multi-factor
authentication. \cite{pender2019parkerian}

Systems with an imperative package manager are more accessible than
declarative systems as learning a new programming language with
esoteric paradigm can pose extra effort. Configuring a whole Nix
system demands a thorough knowledge of Nix language, and that
definitely hinders the ease of access to a Nix system from a system
administrator standpoint. With NixOS, an easy extent of
accessibility can be achieved via planting sufficient configuration
files during device setup.

Atomic systems such as Nix have great benefits towards integrity. As
the "nix store", where every installation is located is read-only,
it's impossible for attackers to modify the store. That's not the
case, where user with root privileges can arbitrarily modify installed
programs and files.

\begin{figure}[t!]
\centerline{\includesvg[width=0.5\columnwidth]{latex/kuvat/ciatriad.drawio.svg}}
\caption{The CIA-triad, a way to demonstrate conflicting security
  measures \cite{hughes2013quantitative}.}
\label{ciatriad}
\end{figure}
\subsection{Security by obscurity}

Currently, NixOS is quite rare in both server and desktop usage as
shown in figure \ref{timeline}. Combined with the unusual file system
and the usage of user-environments, some malware that rely on the
usual locations of installed programs may fail
\cite{nixosSecurityNixOS}.

\subsection{Multi-user installations}
The requirement for root access nearly always widen the potential
attack surface. NixOS provides a way for multiple users installing
their programs through the use of user environments, hence mitigating
the need for root access. This both lessens the availability aspect,
as well as mitigates the programs root access. When file changes are
made in user-specific scope, a thin layer of isolation is
achieved. \cite{nixosNixOSManual}

\subsection{Data integrity}
Data integrity is achieved both by installed programs residing in the
read-only nix store, but also them having been checked against SHA256
checksums. Moreover, the core installation resources for NixOS are
GPG-signed by an administrative Nix team. \cite{nixosSecurityNixOS}
