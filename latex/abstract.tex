
\keywords{tähän, lista, avainsanoista}
% TODO: good/bad keywords

\keywordsen{here, a, list, of, keywords}
\begin{abstract}




Embedded devices are an integral part of our everyday lives; household machines, automotives, and thermal sensors make use of embedded devices at all times. They are subject to the global, developing world's security problems. This thesis focuses on those found in public information screens. Embedded devices are particularly vulnerable to security problems as they have challenges receiving constant, reliable updates. This thesis' focal point is maintaining, updating and upgrading embedded devices. A proposed architecture solution is provided with example snippets to cover most of common security issues found in similar setups. The architecture with it's content is then evaluated through common systems security methodologies. Used methologies are compared to more common methologies. The central theme of this thesis is NixOS, which is a Linux distribution that forms itself from a set of configuration files, supporting features like atomic rollbacks and reliable dependency handling. Most definitive academic sources in this particular subject by Eelco Dostra are used extensively, as well as papers regarding both embedded security and systems security in general. Ideas for further study are presented, as security problems may arise in our everyday lives due to the more mainstream paradigms and could be avoided with the use of declarative ones.

\end{abstract}

\begin{abstracten}
Second abstract in english (in case the document main language is
not english)
\end{abstracten}

