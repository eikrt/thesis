\chapter{Analysis} \label{analysis}

As mentioned in chapter \ref{securitystandpoints} section \ref{queries}, the QuERIES inspired methodology is used in this chapters analysis. Tables \ref{pomdbtable} and \ref{probtable} show relevant information about the calculations. In table \ref{pomdbtable} is listed all the possible states (S), defender actions (D), attacker actions (A) and observations (O). Then, every transition from state to another state was calculated as a probability, shown in table \ref{probtable}. The test counts are seen from the table cells, with their corresponding probability. Note that every row adds up to 1.0.

\section{Modeling the problem and quantifying the models} \label{modprob}

The example project of this thesis is an image showing system that could be used e.g for advertisement, public transport timetables or practically anywhere where static media should be presented. The results, if an an intruder should gain unauthorised access, would be anywhere from displaying improper imagery, to succeeding in displaying propaganda or other unwanted content. Unauthorised access could have a negative economic effect for the service provider, as every organisation displaying media want to remain credible among users.

The attack surface of the example project focuses on physical access and vulnerabilities in remote connections. MQTT-messaging, SSH usage and the usage of display protocols are something where some transfer is happened from one place to another, was it internal or external.

If a unauthorised access would happen, the results would probably affect a part of the society, as the arbitrary content could gain media and social media attention. This public humiliation would definitely affect the credibility of the service provider, as well as the customer. Possible propaganda could affect society by spreading false information, or possibly bringing up societal issues via activism. Either way, this would be unwanted from the perspective of the service provider, customer and possibly the users.

\section{Modeling the possible attacks}

In table \ref{pomdbtable}, the first column describes states the system can be in. The second and the third column state defensive actions, and observations of the system. The fourth column contains template of the attacks that could be conducted. After defining the starting layout, probabilities are calculated based on gathered empirical evidence.

The research setup is a simulated targeted red–blue scenario. The central server is out of the testing scope, as targeting only secure connections in and out from the server is out of the scope for this thesis. For calculating the reward values, R script with library ''pomdp'' was used.
\begin{table}
\centering
\begin{tabular}{ |c|c|c|c|c|c|c|c|c|c| }
\hline
 s&c_t&s_0&s_1&s_2&s_3&s_4&s_5&s_6&s_7 \\
 &&c,p&c,p&c,p&c,p&c,p&c,p&c,p&c,p \\
 \hline
 s_0&&&&&&&&& \\
 \hline
 s_1&&&&&&&&& \\
 \hline
 s_2&&&&&&&&& \\
 \hline
 s_3&&&&&&&&& \\
 \hline
 s_4&&&&&&&&& \\
 \hline
 s_5&&&&&&&&& \\
 \hline
 s_6&&&&&&&&& \\
 \hline
 s_7&&&&&&&&& \\
 \hline

\end{tabular}
\caption{Probabilities of each state transition occurring in a 7 state Markov chain, where s represents state, c count and p probability of the transition succeeding. \(C_t\) represents total count. Given action is ''Monitor state''.}
\label{probtable}
\end{table}
\begin{table}
\centering
\begin{tabular}{ |c|c|c|c|c|c|c|c|c|c| }
\hline
 o&c_t&s_0&s_1&s_2&s_3&s_4&s_5&s_6&s_7 \\
 &&c,p&c,p&c,p&c,p&c,p&c,p&c,p&c,p \\
 \hline
 O_0&&&&&&&&& \\
 \hline
 O_1&&&&&&&&& \\
 \hline
 O_2&&&&&&&&& \\
 \hline
 O_3&&&&&&&&& \\
 \hline
 O_4&&&&&&&&& \\
 \hline

\end{tabular}
\caption{Probabilities of observation across states.}
\label{probtable}
\end{table}
\begin{landscape}
\begin{table}
\centering
\begin{tabular}{ |c|c|c|c| }
 \hline
 & & & &
 S0-S7 & D0-D4 & O0-O4 & A0-A4 \\
 & & & &
 \hline
 \hline
 & & & &
 Idle & Monitor system & Normal operation & Intercept MQTT messaging \\ 
 & & & &
 \hline
 & & & &
 Receive media through MQTT & Patch system & Detected suspicious activity & Compromise Github repository \\  
 & & & &
 \hline
 & & & &
 Setting up SystemD services & Shutdown system & Detected system error & Gain physical access to device \\
 & & & &
 \hline
 & & & &
 Start Weston & Isolate device & Detected unusual media display & Exploit vulnerabilities in display \\
 & & & &
 \hline
 & & & &
 Display media &  & & Exploit vulnerabilities in SSH \\  
 & & & &
 \hline
 & & & &
 Error state & \ & \ & \\
 & & & &
 \hline
 & & & &
 Partial loss of system &  &  &  \\ 
 & & & &
 \hline
 & & & &
 Complete loss of system &  &  &  \\ 
 & & & &
 \hline

\end{tabular}
\caption{Different states, defensive measures, observations and attack measures for the system.}
\label{pomdbtable}
\end{table}
\end{landscape}

A reward function was calculated using the said R program, and the output was x. A weight of 10 was used for positive results, and -100, if something was to be compromised. This weight distribution is due to the fact that even if blue team succeeds most of time, the results of failure are much worse than a succeeding result from the blue team \cite{carin2008cybersecurity}. The discount constant is used in calculations as shown in \ref{pomdpappenix}, where it influences the priority of immediate versus future rewards \cite{mcabeeMarkov}. Our case signifies the importance of both, so value of 0.75 was used.

\section{Using the results}

The results are investigated through the probabilities, which are represented in the table \ref{probtable}. The reward score is taken in to account on how successful/unsuccessful the setup is from a security perspective.

\subsection{Red team implications}

%The result, -30 is negative and that proves that the system has definitive security problems. From the table \ref{probtable} it can be seen that multiple attack vectors succeed in penetrating through the system. The worst probabilities occur when physical access is gained, due to the test device having exposed USB-ports. This weakness could be mitigated, if the production device would have closed USB-ports and a sufficient casing against physical breakthrough. On the other hand, replacing the whole device with a malicious one can't be prevented without the use of external security measures.

\subsection{Blue team implications}

%The blue team tests proved, that the system is very reliable in every state. For example, monitoring the system always worked, and logging could be obtained. Recovering from simulated errors was simple, as the system is very straight forward. A more complex system would perhaps have more issues.

%It's worth noticing that even partial loss of system can be catastrophic, as display servers such as Weston don't need root privileges to operate. This is due security reasons, but comes with the feature that anyone with the user access might display arbitrary content. This is an easy fix, however, because the use of Weston can be denied for all, except for the SystemD user.

% A predefined topology could be used to mitigate device attacks based on device replacing.\cite{fysarakis2014embedded}



%% \begin{table}
%% \centering
%% \begin{tabular}{ |c|c|c|c|c|c|c|c|c| }
%%   \hline
%%   &&&&&&&&& \\
%%   \hline
%%   &&&&&&&&& \\
%%   \hline
%%   &&&&&&&&& \\
%%   \hline
%%   &&&&&&&&& \\
%%   \hline
%%   &&&&&&&&& \\
%%   \hline
%%   &&&&&&&&& \\
%%   \hline
%%   &&&&&&&&& \\
%%   \hline
%%   &&&&&&&&& \\
%% \hline
%% \end{tabular}
%% \caption{Different outcomes of various results gained via QuERIES}
%% \label{probtable}
%% \end{table}
